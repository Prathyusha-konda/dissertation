\chapter{The \emph{Expedition} software suite: Computational tools to analyze transcriptome data}
This is only a test.
\section{A section}
Lorem ipsum dolor sit amet, consectetuer adipiscing elit. Nulla odio
sem, bibendum ut, aliquam ac, facilisis id, tellus. Nam posuere pede
sit amet ipsum. Etiam dolor. In sodales eros quis pede.  Quisque sed
nulla et ligula vulputate lacinia. In venenatis, ligula id semper
feugiat, ligula odio adipiscing libero, eget mollis nunc erat id orci.
Nullam ante dolor, rutrum eget, vestibulum euismod, pulvinar at, nibh.
In sapien. Quisque ut arcu. Suspendisse potenti. Cras consequat cursus
nulla.

\subsection{A Figure Example}
\label{ssec:figure_example}

This subsection shows a sample figure.

\begin{figure}[h] 
  \centering
  \includegraphics[width=0.5\textwidth]{sandiego}
  \caption[Short figure caption (must be \protect{$< 4$} lines in the list of figures)]{A picture of San Diego.  Note that figures must be on their own line (no neighboring text) and captions must be single-spaced and appear \protect\textit{below} the figure.  Captions can be as long as you want, but if they are longer than 4 lines in the list of figures, you must provide a short figure caption.\index{SanDiego}} 
  \label{fig:sandiego}
\end{figure}

\subsection{A Table Example}

While in Section \ref{ssec:figure_example} Figure \ref{fig:sandiego} we had a majestic figure, here we provide a crazy table example.


%%%% TABLE 1 %%%%
\vspace{0.25in}
\begin{table}[!ht]
\caption[Short figure caption (must be \protect{$< 4$} lines in the list of tables)]{A table of when I get hungry.  Note that tables must be on their own line (no neighboring text) and captions must be single-spaced and appear \protect\textit{above} the table.  Captions can be as long as you want, but if they are longer than 4 lines in the list of figures, you must provide a short figure caption.}

\vspace{-0.25in}
\begin{center}
\begin{tabular}{|p{1in}|p{2in}|p{3in}|}

\hline
Time of day & Hunger Level & Preferred Food \\

\hline
8am & high & IHOP (French Toast) \\

\hline
noon & medium & Croutons (Tomato Basil Soup \& Granny Smith Chicken Salad) \\

\hline
5pm & high & Bombay Coast (Saag Paneer) or Hi Thai (Pad See Ew) \\

\hline
8pm & medium & Yogurt World (froyo!) \\

\hline
\end{tabular}
\end{center}
\label{tab:analysis3}
\end{table}


\section{Supplementary Methods}

% For some reason have to add the paragraph indent size here because
% it gets ignored if it's only in the preamble
\setlength\parindent{24pt}


\subsection{Cell culture and differentiation}

iPSCs were cultured on matrigel coated plated using mTeSR (Stem Cell Technologies) media with mTeSR supplement at $37^\circ$ C incubator with 5\% CO$_{2}$.\par

Neuronal progenitor cells (NPCs) were differentiated from iPSCs. Briefly, iPSCs were cultured in Matrigel coated plates and dislodged by dispase. To form embryonic bodies, the dislodged colonies were cultured in DMEM/F12(invitrogen) with GlutaMax and N2 supplement in non-adhere petri dish. Media were replaced every other day for 7 days. EBs were then plated onto matrigel coated plate to allow rosette formation. Clean rosette were picked manually and maintained in EB media for 7 days and subsequently dissociated with accutase and cultured in NPC media (DMEM/F12, GlutaMax, N2 and B27 with \SI[per-mode=symbol]{2}{\micro\gram\per\micro\liter} FGF) to allow neuron progenitor cell differentiation. NPCs were maintained in NPC media.

Motor neurons were directly differentiated from iPSCs as previous described\cite{Chambers:2009ey}. Briefly, iPSCs were cultured on matrigel coated plates until fully confluent in mTeSR then switch to knock-out serum replacement media (KSR) containing Dorsomorphin(\SI{1}{\micro\Molar}) and SB431542 (\SI{10}{\micro\Molar}). Upon day 4 of differentiation, increasing amounts of N2 media (25\%, 50\%) was added to the KSR. From day 7 of differentiation, \SI{1.5}{\micro\Molar} retinoic acid and \SI{200}{\nano\Molar} Smoothened Agonist (SAG, EMD Millipore) were added to induce patterning. Cells were dissociated on day 17 of differentiation and replated in poly-D-lysine and laminin coated plates. Maturation was performed using BDGF (\SI[per-mode=symbol]{2}{\nano\gram\per\micro\liter}), GDNF (\SI[per-mode=symbol]{2}{\nano\gram\per\micro\liter}), CNTF (\SI[per-mode=symbol]{2}{\nano\gram\per\micro\liter}), ascorbid acid, sonic hedgehog and retinoic acid in N2 and B27 media up until 35 days of differentiation.


\subsection{Single-cell capture and library preparation}

iPSCs, NPCs and MNs were dissociated using Accutase (Stem Cell Biotechnologies) and filtered through \SI{40}{\micro\meter} cell strainers to obtain single cell suspension. Single cells were captured on C1 auto prep platform (Fluidigm, CA) according to manufacturer’s instructions. C1 auto prep chips were visually inspected with a light microscopy at 20X to ensure singularity of captured cells. All non-single cells were discarded from analysis. SMARTer Ultra Low RNA cDNA Synthesis Kit (Clontech) was used to reverse transcribe polyA-tailed RNA. cDNA was amplified using Advantage 2 Polymerase Mix by PCR at \SI{95}{\degreeCelsius} for 1 minutes, followed by 21 cycles of 15 seconds at \SI{95}{\degreeCelsius}, 30 seconds at \SI{65}{\degreeCelsius} and 6 minutes at \SI{68}{\degreeCelsius}, followed by another 10 minutes at \SI{72}{\degreeCelsius} as a final extension. cDNAs were inspected using Agilent Bioanalyzer High Sensitivity DNA chips and quantitated by PicoGreen dsDNA Assay kit (ThermoFisher). cDNAs were diluted to \SI{1}{\nano\gram} to generate libraries using the Nextera XT DNA kit (Illumina, La Jolla, CA). Libraries were multiplexed and sequenced on Illumina HiSeq 2000 to generate 100bp PE reads.

\subsection{RNA-Seq processing}

RNA-seq reads were trimmed using \texttt{cutadapt} v1.8.1 of adapter sequences \texttt{TCGTATGCCGTCTTCTGCTTG}, \texttt{ATCTCGTATGCCGTCTTCTGCTTG}, \texttt{CGACAGGTTCAGAGTTCTACAGTCCGACGATC}, \texttt{GATCGGAAGAGCACACGTCTGAACTCCAGTCAC}, $\left[\text{\texttt{A}}\right]_{50}$, $\left[\text{\texttt{T}}\right]_{50}$, mapped to repetitive elements (RepBase v18.05 \cite{Jurka:2005tp}) using the STAR\cite{Dobin:2013fg} splicing-aware aligner (v2.4.01). Reads that did not map to repetitive elements were then mapped to the human genome (hg19), using GENCODE\cite{Harrow:2012cx} (v19) gene annotations to create the splice junction database. We used the  \texttt{SJ.out.tab} files from STAR to create alternative splicing annotations and calcluate percent spliced-in (see Sec.~\ref{sec:outrigger}). Gene expression was quantified with sailfish\cite{Patro:2014jd} using GENCODE v19 protein-coding and long non-coding RNA annotation, and we then aggregated transcript-level expression to genes.

% \subsection{PCR duplicate removal}

% We removed reads which had exactly the same sequence and alignment location using the command \texttt{rmdup} from the \texttt{samtools} suite\cite{Li:2009kaa}. We then indexed the duplicate-removed .bam files with \texttt{samtools index}.

\subsection{Single-cell expression-level quality control and outlier detection}

We retained genes expressed with TPM $> 1$ in at least 10 cells for a total of 18,594 genes, and filtered out cells which had $<4,000$ expressed genes, which was a natural cutoff in the data. For the three cell types, $n=63$ iPSCs, $n=73$ NPCs, and $n=70$ MNs had enough expressed genes to pass gene expression level quality control.

% \subsection{Outlier detection}

We performed $K$-means clustering with $k=3$ on the gene expression matrix, with 1000 different random initializations. For each cell that clustered into a group that consisted of a majority of a different cell type (e.g. a motor neuron that was clustered in the group with majority NPCs), we called these cells outliers and discarded them from analysis. Overall, for iPSC: 71 were captured, 63 passed  QC,  1 outlier for 62 total; for NPC: $98$ were captured, 73 passed QC, 4 outliers for 69 total; for MN: $93$ were captured,  70 passed QC, 10 outliers for 60 total.

% iPSC: 71 were captured, 64 passed  QC,  1 outlier for 63 total; for NPC: 65 + 33 CVN = 98 captured, 76 passed QC, 3 outliers for 73 total; for MN: 63 + 30 = 93 captured,  79 passed QC, 9 outliers for 70 total.


\subsection{Estimation of alternative splicing}
We used \outrigger\, to create a custom alternative splicing index on the splice junction (\texttt{SJ.out.tab}) files created by STAR, and used GENCODE v19 to define possible exons. This created $40,534$ skipped exon (SE) and $13,217$ mutually exclusive exon (MXE) possible alternative events, and we calculated percent spliced-in (Psi/$\Psi$) with a minimum of 10 junction reads. We then filtered for events that were alternative, not constitutively included or excluded across all cells. Alternative events were defined by, $0 < \Psi < 1$, $\Psi \neq 0, 1$ in at least one cell. Events were then filtered for events that were detected in at least 10 cells of any celltype, resulting in $13,910$ events.

% \subsection{Quality control of AS using split single cell libraries}
% To ensure that variations in alternative splicing detection is not due to technical variation in our experimental procedure, we assessed robustness in estimating AS events from single cells by comparing AS events from paired libraries generated from divided single-cells (\textbf{Supplementary Fig.~\ref{fig:splicing_qc}c}). We observed that 60\% of the $15,000$-$18,000$ estimated AS events from the ``splits'' of each cell overlap, with a statistically significant correlation in the percent-spliced-in (Psi/$\Psi$) values (Spearman correlation $R~1$; $p$-value $~0$, \textbf{Supplementary Fig.~\ref{fig:splicing_qc}d}, top row) for the shared AS events. In contrast, when two libraries from distinct single IPS cells were compared, the correlation was high but lower than for the split cells (Spearman correlation $R=0.9$, $p~0$; \textbf{Supplementary Fig.~\ref{fig:splicing_qc}d}, bottom row). Furthermore, the pooled samples’ expression and splicing was highly correlated (Spearman $R=0.94$ and $R=0.97$, respectively; \textbf{Supplementary Fig.~\ref{fig:splicing_qc}e}, bottom row), but compared to a split cell, the expression and splicing correlations were smaller (Spearman $R=0.71$ and $R=0.932$, respectively; \textbf{Supplementary Fig.~\ref{fig:splicing_qc}e}, top row).

\subsection{Constitutive exons}

We defined constitutive exons as those that did not appear as the alternative exon in any of the splice types (MXE and SE), and had at least 10 reads on both upstream and downstream junctions, in at leat 10 cells per cell type.

\subsection{ICA on constitutively expressed genes and their splicing events}
First, $12,685$ genes were identified as non-DE genes across the three populations using a non-parametric Kruskal-Wallis test with Bonferroni-corrected $p$-value, called $q$, with $q > 10$ as the cutoff. 

Second, AS events were extracted from non-DE genes and their Psi scores are subjected to Independent Component Analysis (ICA). To impute the null values widespread in splicing data, we replaced NAs with an arbitrary number (100) out the of range of Psi values. We did not find that the choice of the arbitrary number affected the ICA results. We then calculated ICA on the imputed matrix.

\subsection{Hierarchical clustering}

We performed hierarchical clustering on samples in Python, using the \texttt{fastcluster}\cite{Mullner:2013bl} package and performing optimal leaf ordering\cite{BarJoseph:2001tr} using the \texttt{polo}\cite{Anonymous:2FB4UNR9} package. All clustering was performed using the Euclidean distance metric with Ward's method\cite{WardJr:2012te}. We visualized using the matplotlib\cite{Anonymous:matplotlib} and seaborn\cite{Anonymous:hWlQiCz3} visualization libraries in Python.

\subsection{Gene Ontology Enrichment}
We calculated Gene Ontology (GO) enrichment by using the Gene Ontology mapping queried to the Entrez gene database using the Python package \texttt{mygene}\cite{Wu:2012bo,Xin:2016fv}. We calculated GO enrichment using only the ``biological process'' category, and corrected for multiple hypothesis testing using Bonferroni correction as performed in the Python package \texttt{goatools}\cite{Tang:2015ub}.

\subsection{Categorization of alternative splicing ``modes''}
We calculated modality using the default parameters of the \anchor\, software (see Sec.~\ref{sec:anchor}) only on splicing events observed in at least 10 cells per cell-type. The performance of anchor was tested extensively using simulated data in comparison to existing bimodality detecting methods.

\subsection{Sequence annotation of alternative isoforms}

We annotated alternative events and their biological features at different levels of the Central Dogma.

\subsubsection{DNA-level}

\paragraph{Evolutionary conservation.} We used units of evolutionary conservation as measured by Placental Mammal PhastCons\cite{Siepel:2005cu} scores calculated previously\cite{Lovci:2013cq} (\textbf{Figure~3a-b}, \textbf{Supplementary Fig.~\ref{fig:modality_features}}). 

For average conservation of exons, we used \texttt{bigWigAverageOverBed}\cite{Kent:2010ff} to calculate the mean conservation (treating bases without annotated conservation as NA) across each exon. For base-wise conservation, we used the HTSeq\cite{Anders:2015gf} Python package to create a memory-mapped \texttt{GenomicArray}, and queried this object with the intronic intervals.


\paragraph{Repetitive element overlap.} We used the Repeat Masker track\cite{Rosenbloom:2015bg} from UCSC's Genome Browser\cite{Kent:2002bwa} and used \texttt{bedtools intersect}\cite{Quinlan:2010kma} to overlap with our exon definitions. We grouped repeats into families defined by the Dfam\cite{Hubley:2016fu} database of repetitive DNA elements (\textbf{Supplementary Fig.~\ref{fig:modality_features}e}). For simplicity of interpretation, we used only repetitive elements that appeared at least 10 times in the excluded modality, as it was the modality with the most repetitive elements.

\paragraph{Gene age (Phylostratum)}

We used the Phylostratum classification of genes as found previously\cite{DomazetLoso:2008ba} (\textbf{Figure~3e}). For each splicing event, we found all overlapping genes in the same genomic locus, and aggregated all genes with at least one event in each modality. Meaning, a gene could appear in multiple modality categories if it had one exon in the included modality and another in the bimodal category.


\paragraph{$k$-mer counting and motif (PWM) enrichment}
We used placental mammal conserved elements as downloaded from UCSC\cite{Rosenbloom:2015bg}, taking only conserved elements upstream and downstream of alternative exons. We used \texttt{kvector}\cite{Anonymous:ug} to count $k$-mers in these conserved elements, and calculated a $Z$-score of $k$-mer enrichment for each intron group defined by cell-type, intron context, and modality (\textbf{Supplementary Fig.~\ref{fig:modality_features}k-l}, \textbf{Figure~3d}). Interested in which $k$-mers were enriched in each modality, we used the total $k$-mer counts in the intron context and celltype, for all modalities, as the background. We then performed principal component analysis using the Python package \texttt{scikit-learn}\cite{Pedregosa:2011tv} on the modality introns (\textbf{Supplementary Fig.~\ref{fig:modality_features}m}). We labeled $k$-mers by the standard color of the majority nucleotide (if there was a tie for the winner, the $k$-mer was assigned grey) whose squared PCA distance was greater than two squared standard deviations from the center, i.e. an ellipse around the origin of the plot. We used the Python package \texttt{adjustText}\cite{Anonymous:tk} to move the text labels away from each other and make them readable.

To find which RNA binding protein motifs were enriched for different modalities, we used version 0.6 of the CISBP-RNA binding database\cite{Ray:2013br} and transformed each position-weight matrix (PWM) into a Boolean vector of $k$-mers that could exactly fit into the PWM, with no mis-matches (\textbf{Supplementary Fig.~\ref{fig:modality_features}n-o}). We ignored psuedocounts by setting all values $\leq 0.1$ to zero. We then used this Boolean matrix to obtain motif $k$-mers and calcluate enrichment using a $t$-test, as compared to all $k$-mers of that intron group. We then performed PCA on the motif $t$-statistics, using the intron groups as features (\textbf{Supplementary Fig.~\ref{fig:modality_features}p}, \textbf{Figure~3d}). We labeled motifs whose squared PCA distance was greater than two squared standard deviations from the center, i.e. an ellipse around the origin of the plot. We used the Python package \texttt{adjustText}\cite{Anonymous:tk} to move the text labels away from each other and make them readable.


\subsubsection{RNA-level}

\paragraph{Consistency of splicing between bulk and single-cells}
To calculate the total difference between the bulk $\Psi$ and single-cell $\Psi$ estimates, for each event, we calculated the average difference between the pooled sample $\Psi$ and every single-cell $\Psi$, much like a sample mean calculation (\textbf{Supplementary Fig.~\ref{fig:modality_features}a}).

\paragraph{Splice site strength}
We used \texttt{bedtools}\cite{Quinlan:2010kma} and \texttt{pybedtools}\cite{Dale:2011cl} to obtain the $5^\prime$ (relative to exon-intron boundary: -20nt into intron and +3nt into exon) and $3^\prime$ (relative to exon-intron boundary: -3 into exon and +6 into intron), and obtained the transcript sequences for these regions. We used MaxEntScan\cite{Yeo:2004fg} to calculate the strength of the alternative exon (exon 2 in both the SE and MXE cases) splice sites (\textbf{Supplementary Fig.~\ref{fig:modality_features}f-g}).

\paragraph{Expression of splicing events}
For finding the gene expression per splicing event, for each event, we used all genes that could map to it. Sometimes multiple genes could map to a single event, as a result of poor annotation, or multiple read-through transcripts. To mitigate this, for each event, we summed all gene expression by the $\log_2(\mathrm{TPM}+1)$ values, and plotted the distribution of expression per modality (\textbf{Supplementary Fig.~\ref{fig:modality_features}h}).

\paragraph{Intron and exon length}
As we used \outrigger\, to calculate splicing, it also output the lengths of the introns and exons for each alternative event, which is what we used (\textbf{Figure~3c} and \textbf{Supplementary Fig.~\ref{fig:modality_features}d}).


\subsubsection{Protein-level}

We are in the process of packaging the splicing event isoform translation and domain scanning code into a package called \texttt{poshsplice}\cite{Anonymous:uj}.

\paragraph{Protein translation}
Using events which had at least one isoform annotated with a CDS in GENCODE v19 ($22,152$ SE and MXE events), we translated the exon trio and duo (SE, included isoform has three exons and and excluded has two) or exon trios (MXE, both included and excluded isoforms contain three exons) to its transcript-annotated reading frame. If these exons participated in transcripts with multiple reading frames, we used all translations.

\paragraph{Domain search} We used the \texttt{hmmscan} command from the HMMer\cite{Finn:2011eg,Eddy:1998ut} software suite (v3.1b1) to search for protein domains matching those in the manually curated Pfam-A database\cite{Finn:2016bf}. We used a domain-independent E-value cutoff of $10^{-5}$. With this raw data, we observed ``domain switching'' between isoforms in instances such as ``Kinase'' to ``Tyrosine Kinase'', when indeed the exact characters of domain name changed, but the overall function didn't. To alleviate this problem, we aggregated domains into clades using Pfam's annotations. We then annotated each individual event with whether only the exclusion or inclusion isoforms had an annotated translation, only one isoform, contained a clade, both contained the same clade, or the clades switched (\textbf{Figure~4d}).

\subsection{Correlation of splicing to expression}
We correlated bimodal and multimodal splicing events to genes with variant expression, defined as two standard deviations away from the mean variance of all genes. We used Spearman correlation to compare splicing profiles to gene expression, and used a threshold of absolute correlation values $|R| > 0.5$ across all samples.

\subsection{Transformation of splicing profiles to 2d space}

We used \bonvoyage\, (see Sec.~\ref{sec:bonvoyage}) to transform one-dimensional splicing profiles into two-dimensional space (\textbf{Figure~6a-c}), using the default parameters. We performed the transformation within cell-type, and required at least 10 cells per splicing event to transform.

\subsection{Waypoint-weighted protein properties}

To obtain protein properties, we used IUPRED\cite{Dosztanyi:2005gq} to calculate protein disorder and the \texttt{ProtParam} module in BioPython\cite{Cock:2009hj} to calculate aromaticity, instability index, molecular weight, secondary structure properties (alpha-helix, beta-sheet, and turns), flexibility, grand average of hydropathy (GRAVY) and isoelectric point.



We summarized isoform protein properties for each phenotype by using the NMF-transformed waypoint space into a weighted average. Using $p_{\text{included}}$ and $p_{\text{excluded}}$ to represent the protein property value (e.g. molecular weight or disordered protein score) of each isoform, and $w_{\text{included}}$ and $w_{\text{excluded}}$ to represent the splicing event's waypoint space position for the included ($y$) and excluded ($x$) axes. We calculated the weighted protein property, $p_w$, within each phenotype, as we did for the modality and waypoint calculation.

\begin{align}
p_w = p_{\text{included}} w_{\text{included}} + p_{\text{excluded}} w_{\text{excluded}}
\end{align}

For properties that had a relative center, e.g. isolectric point which has a neutral value of 7, we subtracted the center value for each protein property, $p_{\text{center}}$ so the multiplication by the waypoint space would amplify the distance from center.

\begin{align}
p_w = p_{\text{center}} + (p_{\text{included}} - p_{\text{center}}) w_{\text{included}} + (p_{\text{excluded}} - p_{\text{center}}) w_{\text{excluded}} 
\end{align}

\subsubsection{Voyaging protein properties}

Interested in which protein properties which changed significantly between cell types, we used Mahalonobis distance\cite{DeMaesschalck:2000hv} ($d_m$), a non-parametric method of finding outliers from distributions. In the two-dimensional case, this means values that are significantly ``off-diagonal'' when comparing two cell types, e.g. iPSC to MN. We used a multiplier of $3d_m$ as the threshold for highly changing protein properties.

\subsection{Single-cell qPCR and primer design}

Single iPSCs and differentiated MNs were captured on C1 auto prep platform (Fluidigm, CA). All non-single cells were discarded from analysis. cDNA from single cells were prepared using the Single-Cell-to-Ct kit (ThermoFisher, USA) and pre-amplified with a pool of primers designed for the splicing events and the expression of corresponding genes. Inclusion and exclusion primers were specifically designed to quantitate inclusion and exclusion of AS exons and expression primers were designed from constitutive exons. All primers were tested for amplification efficiency. High-throughout quantitative PCR was performed on 96.96 Dynamic Arrays on BioMark system (Fluidigm) according to manufacturer’s instructions. Each pre-amplified STA sample was diluted 1:15 for iPSCs and 1:10 for MNs. 3 housekeeping genes (RPL22, RPL27, PGK) and lineage genes (POU5F1, LIN28A, DPPA2, ISL1, MNX1, STMN2, NFEL, DCX) were included. The full list of primers is available in \textbf{Supplementary Table 1}.

\subsection{qPCR data processing}

The log expression of each primer set $g$ was computed as $\log(E_{g,c}) = 25 - \mathrm{Ct}_{(g,c)}$ where $c$ is the cell and $\mathrm{Ct}_{(g,c)}$ is the $\mathrm{Ct}$ value for corresponding primer set. iPSCs were filtered by (RPL22 $>5$, LIN28A $> 8$ and POU5F1 $> 8$) and MNs were filtered by (RPL27 $> 9$, ISL1 $> 2$ and STMN2 $> 5$). A total of 134 single iPSCs and 95 single MNs were retained for further analysis. If $\mathrm{Ct}_{\mathrm{xp},c}  > 25$ ($\mathrm{Ct}$ value for the expression primer), the corresponding $\mathrm{Ct}_{(\mathrm{inc},c)}$ ($\mathrm{Ct}$ value for the inclusion primer) and $\mathrm{Ct}_{(\mathrm{exc},c)}$ ($\mathrm{Ct}$ value for the exclusion primer) were excluded from analysis. Percentage of inclusion is calculated by $\frac{2^{\mathrm{Ct}_{\mathrm{inc}}}}{2^{\mathrm{Ct}_{\mathrm{inc}}} +2^{\mathrm{Ct}_{\mathrm{exc}}}}$. Distribution of percentage of inclusion is plotted by violinplot or decomposed into 2-dimension space \texttt{(nmf(dataset, 2, ``lee''))} and projected into waypoint space in R.

\subsection{RNA fluorescence in situ hybridization (FISH)}

To verify alternative splicing of MXE event composed of exon 9 and 10 in PKM, we designed 3 probe sets (Custom Stellaris\textregistered\, FISH Probes, Biosearch Technologies, Inc., CA) using the Stellaris\textregistered\, RNA FISH Probe Designer available online. One set against constitutive exons of PKM labeled with Quasar 570, two probe sets specifically against exon9 or exon 10, respectively, labeled with Quasar 670. For Exon16 SE event in MAP4K4, one probe set against constitutive exons was designed and labeled with Quasar 570 and another probe set against exon16 was designed and labeled with Quasar 670.

iPSCs and MNs grown on coverslip were fixed with 3.7\% formaldehyde PFA for 10 minutes at room temperature. The probes for constitutive (\SI{1.25}{\micro\Molar}) and alternative exons (\SI{1.25}{\micro\Molar}) were mixed and hybridized to the cells in 10\% deionized formamide for overnight at \SI{37}{\degreeCelsius}, according to manufacturer’s instructions. For MNs, a probe set against ISL1 is designed and labeled with fluorescein to allow the counting of only motor neurons. 

\subsection{RNA-FISH image acquisition and data processing}

Images were acquired on Applied Precision OMX Super Resolution System at the Microscopy Core in the School of Medicine. Specifically, transmission and acquisition time were set at 100\% and 2 minutes for both FISH probes (constitutive and alternative exons). DAPI was acquired at 10\% transmission and 20 second to localize the cells. Sections were taken at \SI{0.125}{\micro\meter} for the diameter of the cells, usually around \numrange[range-phrase = --]{10}{12}\SI{}{\micro\meter}. The resulting stacks of images were deconvoluted on Applied Precision OMX workstation. Foci of RNA molecules were quantified using Volocity 6.3 (PerkinElmer). The raw count files were then processed in R to compute ratio of exon inclusion. To limit non-specific foci, only the foci identified by both inclusion probe and constitutive probe were counted for included exons. Normalized inclusion ratio is calculated by percentage of included probes co-localized with constitutive probes/constitutive probes, and resulting percentage is normalized by 95 percent of the maximal percentage.


\section{Supplementary Notes}

\subsection{Bimodal AS events that partition cell populations}

Another example is a bimodal SE event in SUGT1 gene (MIS12 Kinetochore Complex Assembly Cochaperone), encoding a protein involved in kinetochore function and required for the G1/S and G2/M transition. Though alternative variants have been observed, their functions are largely unknown. By clustering global expression with Psi of this event, we identified two distinct subgroups of cells clustered by their Psi score. Noticeably, the subgroup with <0.5 Psi score, indicating exclusion of the alternative exon, demonstrates consistently high expression of ZEB1 (Zinc Finger E-Box Binding Homeobox 1), a master transcription factor regulating epithelial polarity, and was recently reported to be highly expressed in neuron progenitor cells to control neuronal differentiation by repressing polarity genes. Progenitor cells losing ZEB1 expression are likely to exit proliferation and become polarized\cite{Singh:2016iz}. Additionally, this subgroup is enriched with MMP16, reported to be expressed in less differentiated cells\cite{Astarci:2012bk} and a few genes associated with signaling (TSPAN14, involved in presentation of ADAM10, and YES1, a src family tyrosin kinase). In contrast, the other subgroup utilizing the alternative exon highly expresses ERC2 (ELKS/RAB6-Interacting/CAST Family Member 2), encoding a protein actively involved in presynaptic organization of cytomatrix at the active zone (CAZ) complex and function as regulators of neurotransmitter release\cite{Ko:2006gx}, suggesting this subgroup may be on the path to become nascent neurons. Supporting such a possibility, this subgroup is enriched with genes associated with different aspects of neuronal differentiation, such as TBC1D1 (acts as a GTPase-activating protein for Rab family protein(s) involving in vesicle trafficking), ELOVL4 (Very Long Chain 3-Ketoacyl-CoA Synthase 4), EOGT (EGF Domain Specific O-Linked N-Acetylglucosamine Transferase, modifying Notch receptor), FAM60A (Subunit of the Sin3 deacetylase complex (Sin3/HDAC), repressing components of the TGF-beta signaling pathway). Lastly, the two outlier NPCs (demonstrated sufficient coverage of this event and highlighted in grey) presenting higher inclusion of this alternative exon, are projected more towards MNs on PCA (Supplementary Fig 1g) in comparison to the rest of NPCs. Thus, among the NPCs demonstrating bimodality of this SE event in SUGT1, the subgroup with exclusion Psi appears to be more `progenitor-cell' like, whereas the subgroup with inclusion Psi is likely to be geared toward nascent neurons. 



\pagebreak
\section{Software packages}

% From main text: We developed Expedition, a suite of algorithms integrated in a complete software package designed to address three key concepts that are critical for single-cell analyses: (1) rejection of an alternative event if its definition is incompatible with the data, and (2) ability to describe the variation of AS such as detect bimodality and (3) visualize AS distribution changes from one cell type or state to another. 

In this paper, we developed the \emph{Expedition} suite, consisting of  software packages that addressed three key deficiencies in single-cell alternative splicing analysis:

\begin{enumerate}
	\item \textbf{Detect and quantify alternative splicing quickly, with minimum false positives: \outrigger, Section~\ref{sec:outrigger}}\\
	In single-cell analysis, absolute quantitation of gene expression or ``percent spliced-in'' (Psi/$\Psi$) is important and enable us to learn the distribution of these quantitations. Previously, relative quantitation for splicing ($\Delta\Psi$) is more commonly used to calculate the difference between groups. Such relative quantitation tolerates false positive better, as false positives may not vary between groups, $\Delta\Psi \sim 0$ and are thus not noticeable in pairwise comparisons. However, when studying distribution of absolute quantitation, such false positives obsure the observation in unpredictable way and hinder biological interpretation. The second main problem of previous splicing algorithm is the inflexible definitions of alternative exons. The same alternative exons may utilize different flanking exons in different cells/samples, thus leading to different biological interpretation. To address these problems, we create \outrigger, which uses junction reads to find de novo exons, creates a splice graph to define junction-based alternative events, filters for conserved splice sites, and strictly rejectes cases of alternative events incompatible with the data at hand. Finally, we discuss and compare to the popular MISO\cite{Katz:2010iv} algorithm.
	\item \textbf{Classify modalities of alternative splicing events, including bimodal: \anchor, Section~\ref{sec:anchor}}\\
	The power of single-cell analyses rises from the ability to study the distribution of a parameter-of-interest. There are a few statistical methods for finding bimodal distributions, but none are sufficient because they are either not sensitive enough, or not robust enough to noise. Additionally, these methods only deal with bimodal distribution and do not classify other distributions, such as unimodal or multimodal. To create a sensitive distribution classifier for all modalities, we used Bayesian methods to create \anchor, and compare our method to a simple binning method, the bimodality index\cite{Wang:2009wm}, and the bimodal dip test\cite{Hartigan:1985ca}.
	\item \textbf{Quantify and visualize dynamics in distributions: \bonvoyage, Section~\ref{sec:bonvoyage}}\\
	While there are many statistical tests to compare changes in distributions, few of them is coupled with visual tools to present changes in distribution with both magnitude and direction. For the specific question of alternative splicing changes, we are interested in observing a event becomes more included or more excluded. Thus we have employed machine learning methods to create a visualizable, interpretable 2d space with ``included'' and ``excluded'' axes. This method is compared to the quantification offered by the Jensen-Shannon Divergence (JSD)\cite{Anonymous:2011vn}.
\end{enumerate}

\subsection{\texttt{outrigger}: Splicing estimation with \emph{de novo} annotation and graph traversal}
\label{sec:outrigger}

Currently available tools for AS detection and quanitification have two major problems: (1) inflexible definitions that cannot handle different configurations of flanking exons for the same alternative junctions, and (2) lack of rejection of an alternative event even if its definition is incompatible with the data-at-hand. The first problem is solved with \texttt{outrigger index}, which defines all potential alternative events based on the junctions and alternative exons from the aggregate of entire sample sets in a given project, and enumerates all biologically possible flanking exon combinations. This step maximize the likelihood to identify all possible alternative events. To ensure only valid alternative events were generated, we added \texttt{outrigger validate} to remove alternative events with introns lacking conserved splice sites. The second prolbem is solved with \texttt{outrigger psi}, which applies strict rules to only permit junctions with sufficient coverage for an event in a given sample. All the parameters in the rules can be user-defined. Thus, outrigger addresses key issues with current alternative splicing software.

\subsubsection{Algorithm overview}




Broadly, the goal of \outrigger\, is to create a custom, \emph{de novo} alternative splicing annotation by using junction reads and exon definitions to create a exon-junction graph, traversing the graph to find alternative events, and calculate percent spliced-in (Psi/$\Psi$) of the alternative exons.

\paragraph{\texttt{outrigger index}: Create custom alternative splicing annotation.} The following is a narrative describing \textbf{Supplementary Software Fig.~\ref{fig:outrigger_index}a}.

\subparagraph{Inputs.} Two inputs are required for \texttt{outrigger index}: junction counts and gene annotations. The junction counts can be provided in many forms: either \texttt{.bam} \cite{Group:OEYDIUUE} genome alignment files, splice junction count \texttt{.SJ.out.tab} files created by the STAR aligner\cite{Dobin:2013fg}, or a pre-compiled table of samples' junction reads in a \texttt{.csv} format. The gene annotations can be provided in \texttt{.gtf} or \texttt{.gff} format.

\subparagraph{Step 1: Retain junctions from each cell with sufficient read depth.} Junctions with reads in an individual sample less than the minimum number of reads, $r_{\min}$ are removed. By default, $r_{\min} = 10$, and can be adjusted by the user, for example to a minimum of 88 reads, with \texttt{--min-reads~88} on the command line. To illustrate, if one junction is observed with two (2) reads in 100 samples, although there were a total of 200 reads observed on the junction, it will be discarded at this step. Because, there is not sufficient evidence to suggest that this junction is well-covered in any sample. 

\subparagraph{Step 2: Collapse reads on shared exon-exon junctions, across all samples.} The aggregate of all junctions from all samples in a given project are create to maximize the likelihood of identifing all potential alternative events.

\subparagraph{Step 3: Detect exons \emph{de novo}.} If the gap between two junctions is under $X$ nucleotides, an exon will be inserted at the gap. This maximum $X$ is necessary, because otherwise we could insert ``exons'' that are many kilobases long, but aren't true exons -- they are the intergeneic space between genes. By default, $X = 100$, and this can be adjusted by the user, for example to 157 nucleotides, with the command line flag, \texttt{--max-de-novo-exon-length~157}.

\subparagraph{Step 4: Integrate exon annotation to obtain pairwise exon-junction relationships.} Annotated exons are integrated with the \emph{de novo} exons and create a table of the pairwise relationships of each exon to each junction. We do this by creating a database of genes, transcripts, and exons from a GTF gene annotation file using \texttt{gffutils} \cite{Anonymous:sP8uhXuv}, and observing which junctions are adjacent to each exon. This outputs an \emph{``exon-direction-junction''} table which is used in Step 5.

\subparagraph{Step 5: Combine pairwise relationships to obtain global structure.} We then use the adjacencies to build a directional graph which connects exons to each other via junctions. This graph database was built using \texttt{graphlite} \cite{Anonymous:vt}, a Python program that provides a lightweight graph wrapper over SQLite.

\subparagraph{Step 6: Search for alternative exons.} To find alternative events, all exons in the graph database were transversed to test, if starting from that exon, it could be a first exon of an skipped exon (SE) or mutually exclusive exon (MXE) event. 

\subparagraph{Outputs.} The output of \texttt{outrigger index} is a folder containing the following. The \texttt{events.csv} file contains the event definitions will be used by \texttt{outrigger psi}. The \texttt{exonN.bed} files, where \texttt{N} is an exon number, will be used by \texttt{outrigger validate} to check for canonical or non-canonical splice sites.

The splicing event definitions in the \texttt{events.csv} files are specified by the junctions and the alternative exon. As there may be multiple potential flanking exons with the same junctions, rather than choosing a single version (as is done by MISO, \textbf{Supplementary Software Fig.~\ref{fig:outrigger_index}b}), we output all possible flanking exon configurations. Thus, while the critical alternative exons are exon 2 for SE events and exons 2 and 3 for MXE events, we show all possible exon flanking exon 1s and exon 3s for SE, and all possible flanking exon 1s and exon 4s for MXE events (\textbf{Supplementary Software Fig.~\ref{fig:outrigger_index}a}, lower right).

Below is an example command using \texttt{outrigger index}:

\begin{verbatim}
outrigger index --bam *sorted.bam \
    --gtf /projects/ps-yeolab/genomes/mm10/gencode/m10/gencode.vM10.annotation.gtf
\end{verbatim}

This creates a folder called \texttt{outrigger\_output} with the following contents:


\begin{figure}[H]
\dirtree{%
.1 outrigger\_output/.
.2 index.
.3 gtf\DTcomment{Added by Step 3}.
.4 gencode.vM10.annotation.gtf\DTcomment{Added by Step 4}.
.4 gencode.vM10.annotation.gtf.db\DTcomment{Added by Step 4}.
.4 novel\_exons.gtf\DTcomment{Added by Step 3}.
.3 exon\_direction\_junction.csv\DTcomment{Added by Step 4}.
.3 mxe\DTcomment{Added by Step 6}.
.4 event.bed\DTcomment{Added by Step 6}.
.4 events.csv\DTcomment{Added by Step 6}.
.4 exon1.bed\DTcomment{Added by Step 6}.
.4 exon2.bed\DTcomment{Added by Step 6}.
.4 exon3.bed\DTcomment{Added by Step 6}.
.4 exon4.bed\DTcomment{Added by Step 6}.
.4 intron.bed\DTcomment{Added by Step 6}.
.3 se\DTcomment{Added by Step 6}.
.4 event.bed\DTcomment{Added by Step 6}.
.4 events.csv\DTcomment{Added by Step 6}.
.4 exon1.bed\DTcomment{Added by Step 6}.
.4 exon2.bed\DTcomment{Added by Step 6}.
.4 exon3.bed\DTcomment{Added by Step 6}.
.4 intron.bed\DTcomment{Added by Step 6}.
.2 junctions\DTcomment{Added by Step 1}.
.3 metadata.csv\DTcomment{Added by Step 2}.
.3 reads.csv\DTcomment{Added by Step 1}.
}
\caption{Example output of \texttt{outrigger index} command.}
\end{figure}

Besides outputting the relevant \texttt{events.csv} which is used in \texttt{outrigger psi} to define events, we also output \texttt{.bed} files for the entire event, the alternative intron, and each exon, facilitating downstream sequence analysis.


% \subparagraph{Creation of exon-junction graph}  We create a database of genes, transcripts, and exons from a GTF gene annotation file using \texttt{gffutils} \cite{Anonymous:sP8uhXuv}. For each exon, we find all adjacent junctions, that is, all junctions whose \texttt{junction\_start} and \texttt{junction\_stop} are one base downstream or upstream of the exon and add them to our growing table of \emph{``exon-direction-junction''} triples which indicate the pairwise relationships between exons and junctions. We then use this list of adjacencies to build a directional graph.

% \subparagraph{Graph traversal to define alternative events based on junctions} To find alternative events, we traverse all exons in the graph, testing if, starting from that exon, that could be a first exon of an skipped exon (SE) or mutually exclusive exon (MXE) event. We search for SE and MXE events separately. For SE events, first, given an exon, we find all exons downstream. These are potential second and third exons. Second, we get all exons downstream of the potential second and third exons to find potential third exons. We overlap the third exons from the first and second steps to find bona-fide third exons, and traceback to get the second exons. For MXE events, we also start from a potential first exon, then in the second step we get all downstream exons, which are again potential second and third exons. Third, we get the exons downstream of the potential second and third exons to find potential fourth exons. We take overlapping fourth exons from the third step to define MXE events starting the first exon we tested.

% We define the splicing event as simply the junctions and the skipped exon, not the flanking exons, as the junctions define the middle, skipped exons, and allow for the flanking exons to be any length.

\paragraph{\texttt{outrigger validate}: Remove alterantive splicing lacking conserved splice sites.} The following describes the biological intuition behind \textbf{Supplementary Software Fig.~\ref{fig:outrigger_validate_frankenevents}a}. Major (U2) splicesome recognize splice-sites as  ($5^\prime$ end of intron/$3^\prime$ end of intron) \texttt{GT/AG} and \texttt{GC/AG} the Minor (U12) spliceosome recognizes splice-sites as \texttt{AT/AC}\cite{McManus:2011en,GarciaBlanco:2004kl}. By default, these combinations of splice-sties are allowed. But the valid splice sites can be user-specified and changed for example to \texttt{AA/AA} and \texttt{GG/GG} with \texttt{--valid-splice-sites~AA/AA,GG/GG}.

The output of \texttt{outrigger validate} is a \texttt{splice\_sites.csv} folder containing the splice sites, and an additional folder in the splice type folder, called \texttt{validated}, containing filtered \texttt{events.csv} which only contain alternative events with valid splice sites. For example, as a follow up on our previous \texttt{outrigger index} command, we validate the alternative exons with the command,

\begin{verbatim}
outrigger validate --genome mm10 \
    --fasta /projects/ps-yeolab/genomes/mm10/GRCm38.primary_assembly.genome.fa
\end{verbatim}

This creates the following additions to the \texttt{outrigger\_output} folder:

\begin{figure}[H]
\dirtree{%
.1 outrigger\_output/.
.2 index.
.3 gtf.
.4 gencode.vM10.annotation.gtf.
.4 gencode.vM10.annotation.gtf.db.
.4 novel\_exons.gtf.
.3 exon\_direction\_junction.csv.
.3 mxe.
.4 event.bed.
.4 events.csv.
.4 exon1.bed.
.4 exon2.bed.
.4 exon3.bed.
.4 exon4.bed.
.4 intron.bed.
.4 splice\_sites.csv\DTcomment{Added by \texttt{outrigger validate}}.
.4 validated\DTcomment{Added by \texttt{outrigger validate}}.
.5 events.csv\DTcomment{Added by \texttt{outrigger validate}}.
.3 se.
.4 event.bed.
.4 events.csv.
.4 exon1.bed.
.4 exon2.bed.
.4 exon3.bed.
.4 intron.bed.
.4 splice\_sites.csv\DTcomment{Added by \texttt{outrigger validate}}.
.4 validated\DTcomment{Added by \texttt{outrigger validate}}.
.5 events.csv\DTcomment{Added by \texttt{outrigger validate}}.
.2 junctions.
.3 metadata.csv.
.3 reads.csv.
}
\caption{Example output of \texttt{outrigger validate} command.}
\end{figure}


\paragraph{Potential ``Franken-events'' created by combining junctions over multiple datasets.} As many junctions may occur spuriously in a single cell (sample), aggregating all junctions across all cells (sample) may create events that were not observed in any individual cell (\textbf{Supplementary Software Fig.~\ref{fig:outrigger_validate_frankenevents}b}). We wanted to ensure we strictly defined when events were valid or not in these cases.

In the case of SE events, the exon will have $\Psi = \text{NA}$ for the cell with the observed inclusion junctions, since they don't have sufficient reads on both sides of the exon. For the cell with the exclusion junction, it will have $\Psi = 0$ since no inclusion reads were observed.

For MXE events, if each of the four junctions was observed independently in a different cell, then all of the cells will have $\Psi = \text{NA}$ for that splicing event since there are no cells which have sufficient reads on all junctions of either isoform.


\paragraph{\texttt{outrigger psi}: Calculate percent spliced-in of alternative exons}

To calculate percent spliced-in (Psi/$\Psi$) of a potentially alternative exon identified in \texttt{outrigger index}, we use the equation for $\Psi= \frac{\text{inclusion reads}}{\text{total reads}}$ \cite{Wang:2008ea}, with substantial checks for whether the event is valid (\textbf{Supplementary Software Fig.~\ref{fig:outrigger_psi}}). For SE, there is only one exclusion junction and thus the the exclusion junction is weighted by two to compensate (Eq.~\ref{eq:se_psi}). For MXE, the calcluation is simply the inclusion reads divided by the total reads (Eq.~\ref{eq:mxe_psi}). The junction reads between exon $i$ and exon $j$ are presented as $r_{i,j}$, displaying \textcolor{inclusion}{inclusion reads in red} and \textcolor{exclusion}{exclusion reads in blue}.

\begin{multicols}{2}
\noindent
  \begin{gather}
  \text{SE $\Psi$}\nonumber\\
\Psi = \frac{\textcolor{inclusion}{r_{1,2}} + \textcolor{inclusion}{r_{2,3}}}{\textcolor{inclusion}{r_{1,2}} + \textcolor{inclusion}{r_{2,3}} + 2\textcolor{exclusion}{r_{1,3}}} \label{eq:se_psi} %\nonumber
\end{gather}
% \break
\begin{gather}
  \text{MXE $\Psi$}\nonumber\\
\Psi = \frac{\textcolor{inclusion}{r_{1,2}} + \textcolor{inclusion}{r_{2,4}}}{\textcolor{inclusion}{r_{1,2}} + \textcolor{inclusion}{r_{2,4}} + \textcolor{exclusion}{r_{1,3}} + \textcolor{exclusion}{r_{3,4}}} \label{eq:mxe_psi} %\nonumber
\end{gather}
\end{multicols}

Multiple validation steps were incorporated to ensure that the junction reads observed in each sample are consistent with the type of splicing event annotated by \outrigger. This process is described in \textbf{Supplementary Software .~\ref{fig:outrigger_psi}}. 

\begin{description}
	\item[Case 1: Incompatible junctions with sufficient reads.] This step checks whether the junction reads are compatible with a MXE event, or rather a twin cassette event. Specifically, evidence of  $r_{2,3} > r_{\min}$ or $r_{1,4} > r_{\min}$ suggests this junction is a twin cassette event but not an MXE event. In such cases, $\Psi = \text{NA}$. As described in \texttt{outrigger index}, the minimum number of reads is user-defined, for example to 37 with \texttt{--min-reads~37}. 
	\item[Case 2: Zero observed reads.] Given no reads is observed, this event is $\Psi = \text{NA}$, rather than $\Psi=0$ since $\Psi=0$ indicates exclusion. 
	\item[Case 3: All compatible junctions with insufficient reads.] No single junction has the minimum number of reads $r_{\min}$, by default $r_{\min}$ is 10, and can be modifiable by the \texttt{--min-reads} flag. If this is the case, we assign $\Psi = \text{NA}$.
	\item[Case 4: Only one junction with sufficient reads.] This applies to a single junction of two junctions per isoform, e.g. Isoform2 of either SE or MXE events, and Isoform1 of an MXE event, has sufficient reads. Since only one junction has the minimum number of reads, $r_{\min}$, no sufficient evidence indicates inclusion of exon-of-interest, thus, we assign $\Psi = \text{NA}$.
	\item[Case 5: One junction with $>10\times$ more reads than the other.] When the alternative exon is covered on the two sides with junction reads of great disparity, there is insufficient evidence supporting the inclusion of alternative exon or suggests the exon may involved in a complex splicing, rather than a SE or MXE. Thus, $\Psi = \text{NA}$. The default multiplier is 10 and can be modified by the user, for example to 55 by \texttt{--uneven-coverage-multiplier~55}. 
	\item[Case 6: Exclusion: Isoform2 with suffcient reads and Isoform1 with zero reads.] All junctions on Isoform2 have greater than the minimum reads $r_{\min}$, and all junctions of Isoform1 have no observed reads, thus $\Psi = 0$.
	\item[Case 7: Inclusion: Isoform2 with zero reads and Isoform1 with sufficient reads.] All junctions on Isoform2 have no observed reads and all junctions of Isoform1 have greater than the minimum reads $r_{\min}$, thus $\Psi = 1$.
	\item[Case 8: Sufficient reads on all junctions.] Both Isoform1 and Isoform2 have greater than the minimum reads on all their junctions. This is the best possible case for alternative splicing.
	\item[Case 9: Isoform2 with sufficient reads but Isoform1 has one or more junctions with insufficient reads.] If the exclusion isoform, Isoform2 has sufficient reads, but the inclusion isoform (Isoform1) does not, then we assess whether the total read coverage of the event, $\sum_{i,j} r_{i,j}$exceeds $r_{\text{threshold}}$. If so, a $\Psi$ is calculated; if not, $\Psi = \text{NA}$. We define $r_{\text{threshold}}$ as the number of junctions $n$ times the minimum number of reads $r_{\min}$. For example, with a minimum read count is 10 on an SE event, $r_{\text{threshold}} = 30$. For a minimum read count of 10 on an MXE event, $r_{\text{threshold}} = 40$.
	\item[Case 10: Isoform2 has one or more junctions with insufficient reads but Isoform1 has sufficient reads.] Similar to Case 9, we again test if the total read coverage is sufficient to calculate $\Psi$, i.e. if $\sum_{i,j} r_{i,j} \geq r_{\text{threshold}}$. If so, we calculate $\Psi$, and if not, we assign $\Psi = \text{NA}$.
	\item[Case 11: Isoform1 and Isoform2 each have both sufficient and insufficient junctions.] This case only applies to MXE events as SE events have as single Isoform2 junction, and cannot have both sufficient and insufficient junctions. If by the per-junction coverage, it is unclear whether the event has sufficient coverage, then we test if the total coverage of the event is sufficient. If so, we calculate $\Psi$, and if not, we assign $\Psi = \text{NA}$.
\end{description}


% Before we calculate $\Psi$, we ensure that a sample has a valid SE event by checking for enough reads on either both the inclusion junctions ($r_{1,2}$ and $r_{2,3}$) or the exclusion junction ($r_{1,3}$). This protects against calculating $\Psi$ for events that aren't truly SE or MXE events, for example, an alternative first exon event that was annotated as an SE event would have $r_{1,2} = 0$, and thus we wouldn't calculate $\Psi$. For MXE events, we check that there are no reads between exons 2 and 3 ($r_{2,3}=0$), because if there are any reads here, then this is evidence that in a particular sample, this is a twin casette event rather than an MXE event (\textbf{Supplementary Fig.~\ref{fig:outrigger_psi}}). For SE events, we also check that for inclusion, there must be $>10$ reads on both sides of the alternative exon, and if there aren't, then we discard the event.
\subparagraph{Outputs} The output of \texttt{outrigger psi} is added into the \texttt{outrigger\_output} folder by creating a \texttt{psi} folder for each splice type. \texttt{psi.csv} contains $\Psi$ in a matrix, and the \texttt{summary.csv} produces a summary of all the events observed in all samples with their junction reads.

To follow up with our \texttt{outrigger index} and \texttt{outrigger validate} commands, we can run the below example command in the same directory:

\begin{verbatim}
outrigger psi
\end{verbatim}

This command adds to the existing output folder \texttt{outrigger\_output}. Therefore, we don't need to specify a genome location or reads or index location if this command is run from the same folder as the \texttt{outrigger index} command was run, and there exists in the directory a folder called \texttt{outrigger\_output}. 

\begin{figure}[H]
\dirtree{%
.1 outrigger\_output/.
.2 index.
.3 gtf.
.4 gencode.vM10.annotation.gtf.
.4 gencode.vM10.annotation.gtf.db.
.4 novel\_exons.gtf.
.3 exon\_direction\_junction.csv.
.3 mxe.
.4 event.bed.
.4 events.csv.
.4 exon1.bed.
.4 exon2.bed.
.4 exon3.bed.
.4 exon4.bed.
.4 intron.bed.
.4 splice\_sites.csv.
.4 validated.
.5 events.csv.
.3 se.
.4 event.bed.
.4 events.csv.
.4 exon1.bed.
.4 exon2.bed.
.4 exon3.bed.
.4 intron.bed.
.4 splice\_sites.csv.
.4 validated.
.5 events.csv.
.2 junctions.
.3 metadata.csv.
.3 reads.csv.
.2 psi.\DTcomment{Added by \texttt{outrigger psi}}.
.3 mxe\DTcomment{Added by \texttt{outrigger psi}}.
.4 psi.csv\DTcomment{Added by \texttt{outrigger psi}}.
.4 summary.csv\DTcomment{Added by \texttt{outrigger psi}}.
.3 outrigger\_psi.csv\DTcomment{Added by \texttt{outrigger psi}}.
.3 se\DTcomment{Added by \texttt{outrigger psi}}.
.4 psi.csv\DTcomment{Added by \texttt{outrigger psi}}.
.4 summary.csv\DTcomment{Added by \texttt{outrigger psi}}.
}
\caption{Example output of \texttt{outrigger psi} command.}
\end{figure}

\paragraph{Advantages and limitations of \outrigger.}

The main advantages of \outrigger\, are speed and conserved memory footprint. As \outrigger\, operates only on junction reads, rather than resampling reads from a \texttt{.bam} alignment file, which can range in size from 500MB to 20GB and results in a high memory footprint, \texttt{outrigger} summarizes each \texttt{.bam} file to only its junction reads and uses that to estimate Psi/$\Psi$ values. Additionally, employing three steps of \outrigger\, \outrigger\ is able to maximize the number of potential alternative events and subsequently apply strict validation rules in the step of outrigger psi calculation to eliminate false positive events from each sample. 
However, currently, \outrigger\ can only deal with SE and MXE events. We are in the process of incorporating other alternative splice types.

% \subsubsection{PCR duplicate removal studies}

% We also compared outrigger’s performance on pre- and post-duplicate removed dataset. After duplicate-removal, ~93\% events were retained by requiring that each junction is covered by at least 10 unique reads (\textbf{Supplementary Fig.~\ref{fig:splicing_qc}f}). When comparing the Psi scores before and after duplicate-removal, the vast majority of events have a consistent Psi within |delta Psi| < 0.2 and only ~7\% are designated as NA in the latter (\textbf{Supplementary Fig.~\ref{fig:splicing_qc}h-i}), likely due to the reduced read coverage.

\subsubsection{Comparison to other methods}

In comparison to the popular splicing program MISO\cite{Katz:2010iv}, \outrigger\, has three major advantages:

\begin{enumerate}
	\item Ability to build de novo exon indexes (\texttt{outrigger index})
	\item Flexiblity of junction-based definitions of alternative exons, enumerating all possible flanking exons (\texttt{outrigger index})
	\item Ability to eliminate incompatible alternative events (\texttt{outrigger psi})
	\item Speed of evaluation. Instead of using the huge \texttt{.bam} alignment files directly, \outrigger\, summarizes the files as junction reads, leading to much faster calculation of percent spliced-in. Once an index is built with \texttt{outrigger index} (24-48 hours), then calculation of $\Psi$/Psi takes 2-4 hours, even on hundreds of samples. With MISO, the calculation can take 8 hours per sample.
\end{enumerate}


\paragraph{Ability to build de novo exon indexes.} MISO provides pre-built alternative splicing indexes, which may not be incompatible with the data at hand. There is a program, GESS\cite{Ye:2014cd} to detect alternative exons from \texttt{.bam} files, which can only handle a handful files at a time and freeze when given hundreds of single-cell \texttt{.bam} files. In contrast, in the outrigger indexing step, \outrigger\,builds indexes based on provided data, which will be integrated with provided exon annotation allowing identification of novel exons.

\paragraph{Flexiblity of junction-based definitions of alternative exons, enumerating all possible flanking exons.} Multiple possible flanking exons can be associated with an alternative exon, most algorithms, including MISO and rMATS\cite{Shen:2014gs}, choose a single set (often the shortest one), rather than being flexible and allowing the user to choose the relevant ones. The resulting ``best guess'' of the alternative event may not be biologically relavent and may be misleading to interprete. In such case, computational translation of alternative events, as demonstrated in Figure 4, will not be possible. 

\paragraph{Ability to eliminate incompatible alternative events} Comparing MISO $\Psi$ values side-by-side with a corresponding \texttt{outrigger psi} calculation, we find that $46\%$ of MISO $\Psi$ values are rejected and assigned $\Psi = \text{NA}$ by \outrigger\, (\textbf{Supplementary Software Fig.~\ref{fig:miso}}). 

A large group of false positives that are correctly rejected by \outrigger\, are Case 1, where only incompatible junctions present sufficient reads. For example, when twin cassette events are annotated as MXE events and the data indicates inclusion of both alternative exons, MISO will calculate $\Psi$ as 0.5. Because MISO uses a prior of $\Psi=0.5$ and resamples the data to calculate $\Psi$. In such a case, MISO is never convinced that $\Psi$ should be towards 1 or 0 and remains at $\Psi~0.5$ (\textbf{Supplementary Software Fig.~\ref{fig:miso}a}). %As a result, these false positive MXE events make up a far larger proportion of the middle modality when using MISO data than with \outrigger\, (\textbf{Supplementary Software Fig.~\ref{fig:miso}e-f}).


The majority of the false positives are Case 4, where only one junction has sufficient reads. As MISO counts both junctions to calculate $\Psi$, shown in \textbf{Supplementary Software Fig.~\ref{fig:miso}b-c}, many of the events are not covered on both sides of the alternative exons, which may suggest the events are not true SE events, but rather alternative first exon events, for instance. 

We used MISO's event definitions and found that as many as 50\% of MISO events did not pass the stringent rules of \outrigger, primarily due to the incompatibility with the annotation of SE and MXE and insufficient coverage (\textbf{Supplementary Software Fig.~\ref{fig:miso}j-l}). 


% The false positive events identifed by MISO turned out to make up the majority of middle and multimodal modality. Upon visual inspection on IGV track, most of these events are invalid, which prompted us to create \outrigger\ and appling strict rules to reject incompatible events.

% False positives are important to get rid of because a $\Delta\Psi$ could be calculated on false positives, but if in MISO they are both given the value of the prior, then their difference will be near-zero. However, in an aggregate distribution as in \anchor\, or \bonvoyage, all the false positives have the value of the prior, and make it look like there are patterns in the data that aren't truly there.
% # I am not sure this needs to be included.
% \paragraph{Expression biases.} Additionally, MISO is biased such that if a gene is highly expressed, the exon is more likely to be called as included, because its coverage is higher – but, this is not reflective of the fact that the exon is within the ecosystem of the transcript. We downsampled an RNA-seq dataset down by $20\%$ and found that overall, the dataset with more reads had higher $\Psi$ values (\textbf{Supplementary Fig.~\ref{fig:miso}g?}). Exon coverage is a very problematic measure for alternative exon inclusion as this can be biased towards higher inclusion when the entire transcript is simply more expressed. Thus, for alternative exons, the best evidence of inclusion is the exon-exon junction spanning reads that has an intron-sized gap in the alignment and are a hybrid of the exon sequences it spans.


% SE and MXE Psi calculation logic.

% \begin{multicols}{2}
%   \begin{gather}
%   \text{SE $\Psi$ logic}\nonumber\\
% \left( \textcolor{inclusion}{r_{1,2}} \geq r_{\min} \right) \,\mathrm{and}\, \left( \textcolor{inclusion}{r_{2,3}} \geq r_{\min}\right)\label{eq:se_psi_logic}\\
% \mathrm{or}\nonumber\\
% \textcolor{exclusion}{r_{1,3}} \geq r_{\min}\nonumber
% \end{gather}
% \break
% \begin{gather}
%   \text{MXE $\Psi$ logic}\nonumber\\
% \left(\textcolor{inclusion}{r_{2,3}} = 0 \right) \label{eq:mxe_psi_logic}\\
% \mathrm{and}\nonumber\\
% \big[\left( \textcolor{inclusion}{r_{1,2}} \geq r_{\min} \right) \mathrm{and} \left( \textcolor{inclusion}{r_{2,4}} \geq r_{\min}\right)\nonumber\\
% \mathrm{or}\nonumber\\
% \left( \textcolor{exclusion}{r_{1,3}} \geq r_{\min} \right) \mathrm{and} \left( \textcolor{exclusion}{r_{3,4}} \geq r_{\min}\right)\big]\nonumber
% \end{gather}
% \end{multicols}






\subsection{\texttt{anchor}: Modality estimation}
\label{sec:anchor}

\subsubsection{Algorithm overview}
\paragraph{Model modalities as beta distributions}

% \theoremstyle{definition}
We define \emph{modality} as a distinct type of distributions. Since $\Psi$s are continuous value between $(0, 1)$, distribution of $\Psi$ can be modeled as Beta distribution. The probability density function for the Beta distribution, $\mathrm{Pr}(\alpha, \beta)$ is defined between $(0, 1)$, with parameters $\alpha > 0$ and $\beta > 0$,

\begin{equation}
\mathrm{Pr}(\alpha, \beta) \sim \frac{1}{\mathrm{B}\left(\alpha, \beta\right)}  x^{(\alpha - 1)} \left(1-x\right)^{(\beta-1)},
\end{equation}

where $\mathrm{B}\left(\alpha, \beta\right)$ is the Beta function, defined by $\alpha > 0$ and $\beta > 0$. It may be easier to think about how the $\alpha$ and $\beta$ parameters affect distribution by observing the mean and variance \textbf{Supplementary Software Fig.~\ref{fig:anchor_parameterization}a}. The beta distributions can be described by four parameterizations: $1 \leq \alpha < \beta$, $\alpha = \beta > 1$, $\alpha > \beta \geq 1$, $\alpha = \beta < 1$ (\textbf{Supplementary Software Fig.~\ref{fig:anchor_parameterization}b}). Conveniently, these four configurations correspond to the four modalities we are interested in: $1 \leq \alpha < \beta$ corresponds to \emph{excluded}, $\alpha = \beta > 1$ to \emph{middle}, $\alpha > \beta \geq 1$ to \emph{included}, and $\alpha = \beta < 1$ to \emph{bimodal} (\textbf{Supplementary Software Fig.~\ref{fig:anchor_parameterization}c}). The final \emph{multimodal} modality corresponds to $\alpha = \beta = 1$, which is equivalent to the uniform distribution used as null model.
 % in Definition~\ref{def:modality}.

% \begin{definition}{Modality.}
% \label{def:modality}
% A modality is a distinct category of distributions which can be systematically identified.
% \end{definition}


% We are specifically interested in five distinct modalities (Figure~\ref{fig:ideal_modalities}): 

% \begin{description}
%   \item[\0] The majority of the density is near $0$.
%   \item[Middle] The majority of the density is near $0.5$.
%   \item[\1] The majority of the density is near $1$.
%   \item[Bimodal] The density is largely split between the extremes of $0$ and $1$.
%   \item[multimodal] The density does not represent any of the above.
% \end{description}



\paragraph{Model parameterization}
To describe feature distribution as modalities, we parameterized the four parameterizable modalities and used Bayesian model selection to choose the best model to describe the distribution. Python package \texttt{scipy}\cite{Oliphant:2007dm,Millman:2011jv} was used to implement Beta distribution.
% , to nine parameterizations (Figure~\ref{fig:modalities_parameterized}). We found nine parameterizations to be the optimal tradeoff for time and accuracy. In our experiments, the first few parameterizations were often enough to decide on a modality, and we did not find more parameterizations to lead to more accurate estimations. 
For \1 (\0) modality, we fixed $\beta$ ($\alpha$) at 1 and linearly increased $\alpha$ ($\beta$) from $2$ to $20$ (\textbf{Supplementary Software Fig.~\ref{fig:anchor_parameterization}d}). We chose $2$ as a starting parameter since it is near the $\alpha=\beta=1$ uniform distribution, as we wanted to allow \0 and \1 distributions with noise. For bimodal (middle) modality, we changed $\alpha$ and $\beta$ simultaneously, monotonically decreasing (increasing) the parameters from $\alpha=\frac{1}{12}$, $\beta = \frac{1}{12}$ ($\alpha = 2, \beta = 2$) to $\alpha = \frac{1}{30}, \beta = \frac{1}{30}$ ($\alpha = 20, \beta=20$). The parameters for bimodal start at $\frac{1}{12}$ rather than $\frac{1}{2}$ because starting the parameters from $\frac{1}{2}$ resulted in more false positive ``bimodal'' events, whereas starting the parameters from $\frac{1}{2}$ ensures any density near $0.5$ is downweighted.


The fit of feature distribution is assessed to the four configurations using Bayes Factors, represented by $K$,

\begin{align}
K^{(m)} 
&= \frac{P(D | M_1^{(m)})}{P(D | M_0)}\\
&= 
\frac{\sum_{i} P(\alpha_i^{(m)}, \beta_i^{(m)} | M_i^{(m)}) P(D | \alpha_i^{(m)}, \beta_i^{(m)}, M_i^{(m)})}
{\sum P(\alpha_0, \beta_0 | M_0) P(D | \alpha_0, \beta_0, M_0)}\\
&= 
\frac{\sum_{i} P(\alpha_i^{(m)}, \beta_i^{(m)} | M_i^{(m)}) P(D | \alpha_i^{(m)}, \beta_i^{(m)}, M_i^{(m)})}
{1}\\
&= 
\sum_{i} P(\alpha_i^{(m)}, \beta_i^{(m)} | M_i^{(m)}) P(D | \alpha_i^{(m)}, \beta_i^{(m)}, M_i^{(m)})
\end{align}

Where $M_i^{(m)}$ is the model of interest (e.g. $M_i^{(\mathrm{bimodal})}$) and $\alpha_i^{(m)}, \beta_i^{(m)}$ are the corresponding parameters from the parameterization shown in \textbf{Supplementary Software Fig.~\ref{fig:anchor_parameterization}d}. The null model, $M_0$ is the uniform distribution, where $\alpha_0 = \beta_0 = 1$, and thus $P(D|M_0) = 1$ for all datasets. We use a Bayes Factor cutoff of $K_{\mathrm{cutoff}}$ to indicate the threshold where the model begins to explain the data reasonably well. In practice we set $K_{\mathrm{cutoff}} = 2^{5}$ ($\log_2 K_\mathrm{cutoff} = 5$).

The \0 and \1 modalities vary only one parameter at a time, whereas middle and bimodal modalities vary both $\alpha$ and $\beta$ simoutanously. Models with more parameters are more likely to fit, thus we fit to the one-parameter models first, assessing whether $K > K_{\mathrm{cutoff}}$ for either \0 or \1. No distribution can fit both \0 and \1 modalities, thus it is assigned to the modality with highest $K$. Next, the distribution is fitted to the two-parameter bimodal and middle models, checking if $K > K_{\mathrm{cutoff}}$. If neither modality applies, we assign the modality to \emph{multimodal} (\textbf{Figure~2c}). 


As exact $0$ and $1$ are not in the range of the Beta distribution, we implement this model selection by adding a small number ($0.001$) to $0$ and subtracting this small number from $1$. Thus, we approximate the data-derived distribution from the invalid closed interval [0, 1] to the valid open interval of (0, 1).


\subsubsection{Simulations}

We optimized the algorithm parameters using test datasets and visually inspecting random samples from both the best- and worst-fitting data and ensuring that the even the worst fitting data was still believably categorized as the modality (\textbf{Supplementary Software Fig.~\ref{fig:anchor_best_worst}}).

\paragraph{Dataset 1: ``Perfect Modalities'' with noise}
\label{sec:anchor_perfect_modalities}

To test the limits of \texttt{anchor}, we simulated perfectly \0, middle, \1, and bimodal distribution, added uniform random noise with 100 iterations, and estimated modality at each noise level with iteration (\textbf{Supplementary Fig.~\ref{fig:anchor}a}). As expected, the most frequently predicted modality was ``multimodal,'' since the dataset was created from randomly added noise (\textbf{Supplementary Fig.~\ref{fig:anchor}b}). The next frequent modality was bimodal, followed by a tie with excluded and included, and the least frequent one is middle modality. We found that these parameterizations can accurately predict modality with up to $35\%$ noise added to the middle modality, $50\%$ noise added to excluded and included modalities, and up to $70\%$ noise added to the bimodal modality(\textbf{Supplementary Fig.~\ref{fig:anchor}d}). By visual inspection of distributions fit best or worst to each modality (\textbf{Supplementary Software Fig.~\ref{fig:anchor_best_worst}a}), we observed that the bimodal distributions are sufficiently different from other parameterizations, demonstrating the robustness of the algorithm.


\paragraph{Dataset 2: ``Maybe Bimodals'' with noise}
\label{sec:anchor_maybe_bimodals}

To test the proportions of zeros and ones that able to constitute ``bimodal'' distribution, we created another dataset comprised 100 samples of varying amounts of 0s and 1s, and adding random uniform noise (\textbf{Supplementary Fig.~\ref{fig:anchor}h}). The primary predicted modality was bimodal, then multimodal, and finally included and excluded (\textbf{Supplementary Fig.~\ref{fig:anchor}i}). No distribution was predicted as the middle modality, indicating the bimodal and middle modalities are drastically different with little chance of mis-assignment. The falloff of correctly predicting bimodality is at adding $70\%$ noise (\textbf{Supplementary Fig.~\ref{fig:anchor}k}), consistent with the previous simulation with ``Perfect Modalities'' dataset (\textbf{Supplementary Fig.~\ref{fig:anchor}d}). We found that bimodality is determing with a 90:10 (10:90) proportion of samples of 0:1 (0:1) (\textbf{Supplementary Fig.~\ref{fig:anchor}l}). Visual inspection of distributions fit best or worst to each modality confirmed the assignment of each modality(\textbf{Supplementary Software Fig.~\ref{fig:anchor_best_worst}b}).

To summarize, simulation with two different datasets indicates that 1) bimodal modality can tolerate to up to $70\%$ of uniform random noise, and middle modality is least tolerable to noise at only $30\%$, 2) included and excluded modalities are drastically different, so as the middle and bimodal modalities, thus the two step modality assignment procedure (\textbf{Figure 2}) is well-grounded, 3) \anchor is able to determine a bimodal modality with up to 90:10 proportion of zeros and ones.

\subsubsection{Comparison to other methods}

\paragraph{Simple binning} 
We can compare this to other methods we attempted, such as fixing bins of $[0, 0.3, 0.7, 1]$ and using cutoffs for the densities, which does not account for the continuous nature of the underlying distributions. We found the modality whose binned distribution was the smallest distance (measured by Jensen-Shannon Divergence\cite{Anonymous:2011vn}) away from each binned event. In both the simulated modalities and simulated bimodal datasets, we found a sharp increase in multimodal distributions and by eye, poorer categorization of the bimodal modality, especially at the decision boundary of low JSD (\textbf{Supplementary Fig.~\ref{fig:anchor}c, e, j, l, p}).

% \paragraph{Fitting a Beta distribution to individual features}

% or fitting a Beta distribution to each individual feature (which takes a long time) and using cutoffs on the estimated parameters, which is also problematic and error-prone.

\paragraph{Bimodality index}
Another test for bimodality is the Bimodality Index\cite{Wang:2009wm} (BI), which requires estimating each feature as a mixture of Gaussian models. We used the implementation of Generalized Mixture Models in \texttt{scikit-learn}\cite{Pedregosa:2011tv} to estimate two Gaussian distributions for each model, and calculated the BI. For perfect bimodal featues, the value is large, for example, we found that for the zero-noise bimodal event, the $\mathrm{BI}=402$) and was the single bimodality index that was larger than $100$ for any feature (\textbf{Supplementary Fig.~\ref{fig:anchor}f, l, p}). This shows that our method is more sensitive to finding bimodal features with the addition of noise, which BI cannot handle.

\paragraph{Hartigan's Dip test} 
A commonly used test for unimodality is Hartigan's dip test\cite{Hartigan:1985ca}. If the distribution fails the unimodality test, then it is considered bimodal. To define a cutoff for when the dip statistic becomes reliable, we calculated the dip statistic using a Python implementation of the test, called \texttt{diptest}\cite{Anonymous:zTNIPlgQ}. We used a $p$-value cutoff of $p <0.05$ as our threshold for assigning an event as bimodal. We used the diptest statistic on the two datasets, and found that while the zero-noise bimodal event was not detected as bimodal, adding as small amount of noise \emph{improved} the diptest's detection of bimodal events (\textbf{Supplementary Fig.~\ref{fig:anchor}g, m, q}), and the accuracy dropped off at a very high noise level - 90\%. As expected, the excluded, included, and middle modalities weren't detected as bimodal, except at higher noise levels, which we also saw with \anchor.


\subsection{\texttt{bonvoyage}: Transformation of distributions to \emph{waypoints} and \emph{voyages}}
\label{sec:bonvoyage}

\subsubsection{Algorithm overview}

The goal of \bonvoyage\, is to be able to summarize the entire distribution of a feature into a single point in space, enabling visualization multiple distributions at a time with intuitive interpretation. To accomplish this, we will transform one-dimensional vectors into two-dimensional space. Specifically, the $x$-axis will represent the \emph{excluded} dimension and the $y$-axis will represent the \emph{included} dimension, and all points will be described as a sum of \0 and \1 components (\textbf{Figure~6a}, left). For example, for two distinct cell-types, we can imagine a feature that starts at a \1 modality in the first and changes to a \0 event in the second, or changes from middle to bimodal (\textbf{Figure~6a}, right). 



\paragraph{Data discretization}
We will use a reduced representation of our splicing data by binning each feature on bins $b$ of size $0.1$, where $b_n$ represents the $n$th bin. We represent the binned splicing matrix with $B_\Psi$, where $B_\Psi[k,j]$ represents the fraction of non-null samples in feature $j$ with $\Psi$ value contained in $b_k$. In practice, we pre-filter the data by using only features for which there are enough samples. In the main text for this paper, we used a minimum of 10 cells.

\paragraph{Dimensionality reduction via non-negative matrix factorization}

Non-negative matrix factorization (NMF) is a parts-based dimensionality reduction algorithm which results in meaningful, interpretable results \cite{Lee:1999gw}. It is an alternative to other dimensionality reduction methods such as principal- and independent- component analyses (PCA and ICA) because its features are both independent, and non-negative, and thus each feature is composed of a sum of the underlying structure of the data, without pesky negative terms.

Thus, for NMF, we will be reducing $B_\Psi$ as such,

\begin{equation}
B_\Psi \approx W \times H,
\end{equation}

Where $W$ is a (features, $2$)-size matrix of the composition of each feature as a sum of how many samples are excluded and included. We found that in the alternative splicing data, the primary components were the included and excluded values, but in other datasets, this may not be the case. Thus, as the components of NMF are the most prominent features, to ensure reproducibility of the axes across datasets, we seeded the NMF transformation with a matrix that is composed of features that are primarily \0 plus a single \1 feature (\textbf{Supplementary Fig.~\ref{fig:bonvoyage}a}). We used the Python package \texttt{scikit-learn} \cite{Pedregosa:2011tv} for the Projected Gradient NMF implementation. 

We call the projected distributions ``waypoint space,'' and the distance between two points a ``voyage,'' such as the voyage of the MXE event in PKM (\textbf{Figure~6c}). 

% For multiple cell-types, we as we will show in the simulations (the next section, \ref{subsubsec:bonvoyage_simulations}), we \emph{could} plot all voyages as arrows, but for many at a time, it can be easier to visualize through a hexagonally binned scatterplot, where the $x$-axis is the $\Delta$excluded axis, and the $y$-axis represents the $\Delta$included axis.

\subsubsection{Simulations}
\label{subsubsec:bonvoyage_simulations}

% The goal of \texttt{bonvoyage} is to identify features which change across groups. As diagrammed in Figure~\ref{fig:example_feature}, the idea is to transform distributions of values into a single point onto \emph{waypoint space}, find the distances between transformed distributions and plot the vectors onto \emph{voyage space}.

\paragraph{Transformation of static distributions}

To demonstrate the ability of \texttt{bonvoyage}, we created a simulated dataset which we call ``Maybe Everything'' consisting of every combination of 0s, 1s, and 0.5s (\textbf{Supplementary Fig.~\ref{fig:bonvoyage}a-d}), essentially incorporating both the ``Perfect Modalities'' (from \ref{sec:anchor_perfect_modalities}) and ``Maybe Bimodals'' (from \ref{sec:anchor_maybe_bimodals}) into a single dataset. Again, we added uniform random noise at $5\%$ intervals. We transformed the entire simulated dataset into the \emph{``waypoint''} space.

% For visualization of the two-dimensional \emph{waypoints}, we use a binning approach rather than a scatterplot, as it allows the user to visualize the density more readily. We use hexagonal binning as hexagons have the unique property of being the largest polygon which allows for regular tesselation of two-dimensional space, allowing for efficient visualization of density in two dimensions \cite{Carr:2012hi}. Specifically, we see that with the noisiest dataset, the density is highest in the middle of the triangle.

% To check whether our NMF projection correctly matches our modalities, we plot the waypoints as a scatterplot and paint the features by modality color as assigned by \texttt{anchor} (\textbf{Supplementary Fig.~\ref{fig:bonvoyage}E}). For this visualization, we chose a scatterplot as the hexagonal bins of different modalities may overlap and thus not faithfully represent the true data.

% \paragraph{Detecting changes in distribution}

To identifying features which change in distribution, we calculate the \emph{``voyage"} between them in waypoint space. As a demonstration, we shuffle the simulated data to create two different \emph{in silico} phenotypes. We will use each feature as a \emph{``waypoint"} along the voyage, and calculate total travel distance of each feature between the phenotypes. %In Figure~\ref{fig:voyages_histogram}, we show the histogram of the total distance traveled across the three simulated phenotypes.


A key aspect of the waypoint space is that while changes from exclusion to inclusion are easy to spot by a change in means, the change from a middle to a bimodal is not, and requires a battery of other tests to find. Here, voyage space has a significant advantage as it gives both the magnitude of change and a directly interpretable direction. %Here, voyage space has the advantage of presenting both the magnitude of change and the direction, and an event which changes from middle to bimodal will have the ``north west" direction.


% We break up the voyage space into four quadrants, corresponding are four possible directions of change for a feature, as shown in Figure~\ref{fig:travel_directions_cartoon}. However, not all directions have equal opportunity for movement. In particular, the features which travel northwest (towards bimodal) or southeast (towards middle), have a smaller distance scale than those that travel northeast (towards inclusion) or southwest (towards exclusion), as shown in the histograms in Figure~\ref{fig:voyages_histogram}.

% Again, to facilitate visualization of the voyages, we employ a hexagonal binned plot, which indicates the areas of highest density, as shown in Figure~\ref{fig:voyages_scatter_hexbin}. In Figure~\ref{fig:voyages_transitions_facetgrid}, we show the voyages of features whose distance is larger than the mean (cutoff shown in Figure~\ref{fig:voyages_histogram}), separated on a per-transition basis.



\subsubsection{Comparison to other methods}

As there exist many methods for comparing distributions, we will show that the magnitude of change obtained from \texttt{bonvoyage} is comparable to other metrics for assessing changes in distribution. In particular, we will show the metrics within each modality, and across modalities, compared to Jensen-Shannon Divergence\cite{Anonymous:2011vn} (JSD) in (\textbf{Supplementary Fig.~\ref{fig:bonvoyage}e}). While JSD is more sensitive to slight changes in distribution (their scatterplots are skewed towards the right), it does not also encode directionality of change. Thus, \texttt{bonvoyage} offers a unique perspective on how to interpret changes in distribution.



\section{Supplementary Software Figures}

% All figures are Supplementary
\renewcommand{\figurename}{Supplementary Software Fig.}


% This line is important as it increments the figure count and keeps the figure numbers correct
% For whatever reason, the caption of Figure 1 is too big and cannot be put into a "minipage" and
% thus doesn't get an incremental counter.
\setcounter{figure}{0}

\subsection{Supplementary Software Figure 1}

\begin{minipage}{\textwidth}

% \centering
\captionof{figure}{\textbf{Examples of inconsistencies in MISO's estimation with single-cell data.}\\
% \textbf{Supplementary Fig. 1: Examples of inconsistencies in MISO's estimation with single-cell data.}\\
\textbf{a-c. Representative examples of SE and MXE AS events measured by MISO, but were unsupported with visual inspection on IGV browser, and were disqualified by \outrigger.} To identify SE and MXE events, \outrigger\, constructs a \emph{de novo} splicing index based on the junction reads in all libraries in the dataset (see details in \textbf{Supplementary Software Figs. \Cref{fig:outrigger_index,fig:outrigger_psi,fig:outrigger_validate_frankenevents}}). The following examples are not considered bys \outrigger as true SE or MXE events, therefore annotated as NA. Note, MISO does not estimate modality for each event, \anchor\, (see details in \textbf{Supplementary Fig.~\ref{fig:anchor}}) was used to estimate modality.\\
\textbf{a.} Top, a MISO-annotated MXE event in ARF4 with MISO estimated $\Psi$s $\sim0.5$ and classified as ``middle'' modality in each of iPSC, NPC, and MN by \anchor. Yet, in the IGV browser (bottom), this event appears as a twin cassette event, where both exons 2 and 3 are included, indicating that at least in our dataset this event is not consistent with the MISO annotation. Outrigger disqulifies this event as a MXE and assign NA (top left).\\
\textbf{b.} Top, a MISO-annotated SE event in CLF1 with MISO estimated $\Psi$s ranging from $0.1$ to $0.6$ and is classified as a ``middle'' modality event by \anchor\, in each of iPSC, NPC, and MN.  Yet, in the IGV browser (bottom), exon 1 for this annotation is not covered at all. Given the data, outrigger\ do not consider this as a bona fide SE event and assign NA to this event.\\
\textbf{c.} Top, a MISO-annotated MXE event in AHSA1 with a wide range of MISO calculated $\Psi$s and is classified as the ``multimodal'' modality in each of iPSCs, NPC, and MN populations by \anchor. Bottom, in the IGV browser. Exons 2 and 3 are the annotated alternative exons for MXE, however, another two well-covered exons between exon 2 and 3 were observed and one extra exon between exon 3 and 4, which disqualify this event as an MXE event. Furthermore, when both exon 2 and 3 are included, MISO estimated $\Psi$ scores are closer to 1 instead of around $0.5$, as was seen in (\textbf{a}). Thus, outrigger rejects this as MXE and assign NA.\\
\textbf{d.}~Using \outrigger\,'s strict rules on MISO annotations, the majority (51\%) of the data generated by MISO was rejected by \outrigger\, (left). Right, using the exact same annotation from MISO, \outrigger\, 22\% of events found by \outrigger\, had too wide of a confidence interval ($>0.4$) by MISO.\\
\textbf{e.}~Heatmap comparing the numbers and percentages of alternative events that were within $|\Delta\Psi| < 0.2$, switched to exactly 1 or 0 in \outrigger, were NA in either MISO or \outrigger, or were in another case.\\
\textbf{f.}~Barplot of the number of cases found only in MISO (orange) and rejected as NA by \outrigger, and of the cases found only by \outrigger (green) and considered to have too wide of a confidence interval by MISO.\\
% \textbf{(d-e). Comparison of MISO and \outrigger\, in description of gobal landscape of AS in single cells.} Due to the discrepancies between MISO's prior of $\Psi=0.5$ and annotations in our single-cell data, the distribution of $\Psi$ and the number of events categorized to each modality differ and lead to conflicting interpretations.\\
% \textbf{d.} Top, lavalamp plot of 12,894 SE and MXE splicing events identified by \outrigger. Bottom, lavalamp plot of 19,662 splicing events identified by MISO, with a confidence interval $<0.4$. MISO identifies more events, however, many of them appeared to be mis-identified as shown in examples (\textbf{a-c}). Additionaly, MISO estimated $\Psi$s tend to be distributed more boardly in comparison to \outrigger's, probably due to MISO's prior of $\Psi=0.5$.\\
% \textbf{d.} Left, heatmap of the number of events found in each modality, for each celltype, by MISO. Right, heatmap of the number of events found in each modality, for each celltype, by \outrigger. MISO identifies one to three orders of magnitude more events in both middle and multimodal modalities, most likely due to MISO's prior of $\Psi=0.5$ and discrepant annotations.\\
% \textbf{e.} Top, the percentage of exons in each modality (left) and the proportion of SE and MXE events (right) that were identified by \outrigger. Bottom, the percentage of exons in each modality (left) and the proportion of SE and MXE events (right) that were identified by MISO. MISO tends to identify more MXE events or events assigned as ``middle'', probably due to the inconsistency of MISO annotation with our datasets as illustrated in (\textbf{a-c}). MISO also identifies more events assigned as ``multimodal'' modality likely due to a bias against perfect inclusion ($\Psi=1$) or exclusion ($\Psi=0$), resulting in fewer events with $\Psi=1$ and $\Psi=0$ it calculated as illustrated in (\textbf{c}).\\
To summarize, \outrigger\, follows strict rules to identify alternative splicing (\textbf{Supplementary Software \Cref{fig:outrigger_index,fig:outrigger_psi,fig:outrigger_validate_frankenevents}}) and provides a $\Psi$ distribution more localized at the extremes of $\Psi=0$ and $\Psi=1$. Although \outrigger\, may identify fewer events, they are true SE and MXE events.
\label{fig:miso}
}
\end{minipage}



\subsection{Supplementary Software Figure 2}

\begin{minipage}{\textwidth}
\centering
\captionof{figure}{\textbf{Internal steps of indexing via \texttt{outrigger index}: Exons identification and defining alternative events.} \\
\textbf{a.} Internal workings of the indexing step via \texttt{outrigger index}. User-provided inputs junction reads can be either genome-aligned \texttt{.bam} files, the \texttt{.SJ.out.tab} splice junction files from the STAR aligner, or a compiled table in \texttt{.csv} of all junction reads from all samples for the project. Step 1, only junction reads with sufficient depth in a cell/sample are retained. By default, the minimum number of reads is $10$ per cell/sample, which can be modified with the flag \texttt{--min-reads}. Step 2, junction reads are used to identify junction locations, and reads are aggregated across all cells/samples regardless of which cell/sample it came from. Step 3, if there is a ``gap'' between two junctions that is smaller than certain length $X$ (by default, $X=100$ nucleotides but can be modified with the flag \texttt{--max-de-novo-exon-length}), then an exon is inserted. Step 4, the identfied exons are compared with the annotated exons to obtain the pairwise relationships between exons and junctions. Step 4 outputs a table of ``triples:'' of \texttt{(exon, direction, junction)} encoding the directional relationship between exons and junctions. Step 5, the output tables from step 4 are utilized to connect exons through junctions and creates a graph database. Finally, in Step 6, alternative exons are identified by traversing the graph database. The output of the indexing step run by the command \texttt{outrigger index}, is junction-based, outputting the alternative exon and all possible configurations of flanking exons for each event. For example, on the bottom right, the same skipped exon event using the same alternative junctions, have four possible configurations of flanking exons. They are considered to be the same event, but are reported with all four configurations for the ease-to-use in downstream analysis.\\
\textbf{b.} Defining alternative events and comparison of biological interpretability of events found by MISO and \outrigger. For a given alternative exon (black box), there can be multiple transcripts corresponding to the alternative exon but with different flanking exons. MISO chooses to define the alternative event using the shortest exons on both sides. Yet, this MISO-defined alternative event may not actually exist as a transcript in the dataset and will be misleading to interpret. For example, attempts to translate such non-existing transcript(s) will be inappropriate. In contrast, \outrigger\, defines the event based on the junctions, and outputs all corresponding flanking exon configurations, thus enabling boader use of the outputs and more relevant biological interpretation. 
}
\label{fig:outrigger_index}
\end{minipage}



% \subsection{Supplementary Software Figure 2}
% \label{sec:sfig6}
% \begin{minipage}{\textwidth}
% \centering
% \captionof{figure}{\textbf{Validation via \texttt{outrigger validate}: Removal of alternative events with introns lacking consensus splice sites.} \\
% In this optional step, exons with flanking introns lacking known splice site motifs are removed. This is configurable. By default, the valid splice sites are specified as, \texttt{--valid-splice-sites GT/AG,GC/AG,AT/AC}, but can be any pair of two nucleotides.}
% \label{fig:outrigger_validate}
% \end{minipage}

\subsection{Supplementary Software Figure 3}

% \begin{figure}[H]
\begin{minipage}{\textwidth}
\centering
% \includegraphics[width=\textwidth]{\maintext/Figure S2.pdf}
\captionof{figure}{\textbf{\outrigger\, validation and pathological cases.} \\
\textbf{a.}~Validation via \texttt{outrigger validate}: Removal of alternative events with introns lacking consensus splice sites. In this optional step, exons with flanking introns lacking known splice site motifs are removed. This is configurable. By default, the valid splice sites are specified as, \texttt{--valid-splice-sites GT/AG,GC/AG,AT/AC}, but can be any pair of two nucleotides.
\textbf{b.}~Possible pathological cases of \texttt{outrigger}. These ``Franken-events'' consist of junctions that were observed in independent samples. At the indexing step, aggregated reads from multiple cells/samples are considered to construct an index of all junctions to maximize the number of AS events. Yet, at the Psi/$\Psi$ calculation step, in each individual cell/sample, insufficient reads may be observed for certain junction resulting in $\Psi=\text{NA}$ in some cells/samples for the same event. Top, skipped exons, if each junction is observed only in one cell, the cell with the exclusion junction is assigned a $\Psi=0$ while the remaining cells are assigned as $\Psi=\text{NA}$. Bottom, mutually exclusive exons, $\Psi=\text{NA}$ for all 4 cells, as there is insufficient evidence of exon inclusion or exclusion in any one cell. Thus, the number of detected events output by \texttt{outrigger index} can greatly overestimate the number of valid events in the dataset found by \texttt{outrigger psi}.}
\label{fig:outrigger_validate_frankenevents}
\end{minipage}



\subsection{Supplementary Software Figure 4}

\begin{minipage}{\textwidth}
\centering
\captionof{figure}{\textbf{Cases created by percent spliced-in calculation via the command \texttt{outrigger psi}.} The table describes the 11-step sequential logic of \outrigger\, to reject an event in a cell/sample based on that cell/sample's junction reads. If an event reaches a $\Psi=\text{NA}$ case, then it is rejected from that sample, otherwise, it continues through the cases. If the event is rejected, then it is assigned $\Psi = \text{NA}$, if it is not rejected, then it gets a $0\leq \Psi \leq 1$ value based on the junction reads.\\
Strict evalution of percent spliced-in (Psi/$\Psi$). To compute the percent spliced-in (Psi/$\Psi$) of skipped exon (SE) and mutually exclusive exons (MXE) alternative events during the execution of the command \texttt{outrigger psi}, we use $\Psi= \frac{\text{inclusion reads}}{\text{total reads}}$. We represent the number of reads spanning the junction between exon$_i$ and exon$_j$ as $r_{i,j}$.
% \textbf{g.}~Left, Model of exons, junctions and reads for an SE event. Right top, logic of whether to calculate $\Psi$ for an event in a sample. for SE, to avoid mis-annotated events, we check for the following logic to be fulfilled, specifically require for \emph{both} inclusion reads to be greater than the minimum, or for the exclusion junction read number to be greater than the minimum reads, as shown. Right bottom, we then calculate $\Psi$ as shown.\\
% \textbf{h.}~Left, Model of exons, junctions, and reads for an MXE event. Right top, logic of whether the junctions in a sample are consistent with an MXE event. For a particular sample there are one or more reads between exons 2 and 3 ($r_{2,3} > 0$), then this is evidence for a twin cassette event, not an MXE event, then for this sample $\Psi$ is not calculated for this event. To avoid improperly annotated events, we use only events with enough reads on all inclusion or exclusion junctions, as shown. Right bottom, we then calculate $\Psi$.
}
\label{fig:outrigger_psi}
\end{minipage}






\subsection{Supplementary Software Figure 5}
\begin{minipage}{\textwidth}
\centering
\captionof{figure}{{\bf Overview of \anchor\, parameterization of the Beta distribution.} \\
\textbf{a.}~Top, equation for the Beta distribution of the random variable $x$ with parameters $\alpha, \beta > 0$. Bottom left, equation for the mean ($\mu$) of the Beta distribution as a function of its parameters. Bottom right, equation for the variance ($\sigma^2$) of the Beta distribution as a function of parameters.\\
\textbf{b.}~Cartoon of valid values of $\alpha$ and $\beta$ parameters of Beta distribution, showing how the space is partitioned by the modalities.\\
\textbf{c.}~Violinplots representing the four ideal modalities, plus the null ``multimodal'' distribution. Each modality is annotated with examples of four cells representing within-cell distributions of included (dark grey) and excluded (light grey) transcripts, and the corresponding parameters of the Beta distribution.\\
\textbf{d.}~Violinplots of 1 million random samples of the family of Beta distributions specified by the $\alpha$ and $\beta$ ($x$ tick labels) parameterization of the four modalities: excluded, bimodal, included, and middle.}
\label{fig:anchor_parameterization}
\end{minipage}


\subsection{Supplementary Software Figure 6}
\begin{minipage}{\textwidth}
\centering
\captionof{figure}{{Best and worst fitting modality data using \anchor.} \\
Left, 10 events with largest Bayes Factor, $K$ (best fit) from the assigned modality. Right, 10 events with smallest Bayes Factor, $K$ (worst fit) from their assigned modality. For multimodal, as there is no fit, this simply shows 20 random events.\\
\textbf{a.}~Bayesian \anchor\, method on ``Perfect modalities'' dataset.\\
\textbf{b.}~Bayesian \anchor\, method on ``Maybe bimodals'' dataset.\\
% \textbf{c.}~Binned method on ``Perfect modalities'' dataset.\\
% \textbf{b.}~Binned method on ``Maybe bimodals'' dataset.\\
}
\label{fig:anchor_best_worst}
\end{minipage}






\bibliographystyle{plainnat}
\bibliography{refs}
