%
%
% UCSD Doctoral Dissertation Template
% -----------------------------------
% http://ucsd-thesis.googlecode.com
%
%


%% REQUIRED FIELDS -- Replace with the values appropriate to you

% No symbols, formulas, superscripts, or Greek letters are allowed
% in your title.
\title{Computational analysis of single-cell alternative splicing}

\author{Olga Borisovna Botvinnik}
\degreeyear{\the\year}

% Master's Degree theses will NOT be formatted properly with this file.
\degreetitle{Doctor of Philosophy}

\field{Bioinformatics and Systems Biology}
%\specialization{Anthropogeny}  % If you have a specialization, add it here

\chair{Professor Gene Yeo}
% Uncomment the next line iff you have a Co-Chair
\cochair{Professor Sheng Zhong}
%
% Or, uncomment the next line iff you have two equal Co-Chairs.
%\cochairs{Professor Chair Masterish}{Professor Chair Masterish}

%  The rest of the committee members  must be alphabetized by last name.
\othermembers{
Professor C. Titus Brown\\
Professor Amy Pasquinelli\\
Professor Sam Pfaff\\
Professor Kun Zhang\\
}
\numberofmembers{6} % |chair| + |cochair| + |othermembers|


%% START THE FRONTMATTER
%
\begin{frontmatter}

%% TITLE PAGES
%
%  This command generates the title, copyright, and signature pages.
%
\makefrontmatter

%% DEDICATION
%
%  You have three choices here:
%    1. Use the ``dedication'' environment.
%       Put in the text you want, and everything will be formated for
%       you. You'll get a perfectly respectable dedication page.
%
%
%    2. Use the ``mydedication'' environment.  If you don't like the
%       formatting of option 1, use this environment and format things
%       however you wish.
%
%    3. If you don't want a dedication, it's not required.
%
%
\begin{dedication}
  To my family, my parents, and Kwasi.
\end{dedication}


% \begin{mydedication} % You are responsible for formatting here.
%   \vspace{1in}
%   \begin{flushleft}
% 	To me.
%   \end{flushleft}
%
%   \vspace{2in}
%   \begin{center}
% 	And you.
%   \end{center}
%
%   \vspace{2in}
%   \begin{flushright}
% 	Which equals us.
%   \end{flushright}
% \end{mydedication}



%% EPIGRAPH
%
%  The same choices that applied to the dedication apply here.
%
\begin{epigraph} % The style file will position the text for you.
  \emph{Always stay gracious, best revenge is your paper -- Beyonc\'e Giselle Knowles Carter}
%   ---Smarty Pants
\end{epigraph}

% \begin{myepigraph} % You position the text yourself.
%   \vfil
%   \begin{center}
%     {\bf Think! It ain't illegal yet.}
%
% 	\emph{---George Clinton}
%   \end{center}
% \end{myepigraph}


%% SETUP THE TABLE OF CONTENTS
%
\tableofcontents


%%% --- BEGIN trying stuff from stackoverflow --- %%%
% For some reason doing these modifications in ucsd.cls doesn't do anything
% so I'm forcing them to happen here and modify only the \listoffigures command
{
\let\oldnumberline\numberline
\renewcommand*{\numberline}[1]{%
  \oldnumberline{\figurename~#1:}}%
% {\ssp\@starttoc{lof}}%
% \if@restonecol\twocolumn\fi%
% \renewcommand{\numberline}{\figurename~\oldnumberline}
\listoffigures  % Comment if you don't have any figures
}
%%% --- END trying stuff from stackoverflow --- %%%
\listoffigures
% \listoftables   % Comment if you don't have any tables


%% ACKNOWLEDGEMENTS
%
%  While technically optional, you probably have someone to thank.
%  Also, a paragraph acknowledging all coauthors and publishers (if
%  you have any) is required in the acknowledgements page and as the
%  last paragraph of text at the end of each respective chapter. See
%  the OGS Formatting Manual for more information.
%
\begin{acknowledgements}

There are too many people to thank but I will do my best. 

Thank you to Dr. Yan Song for being a patient mentor, keeping me on track, and giving key criticisms of the internal workings of all the algorithms, and for making sure everything is explained as clearly as possible. I couldn't have done any of this without you.

Thank you to Dr. Gene Yeo for giving me freedom and opportunities to teach my passions of open-source software, open science to unsuspecting biologists.

Thank you to Dr. Sheng Zhong for key computational feedback on \texttt{anchor}'s classifier.

Thank you to Dr. C. Titus Brown for believing in me and sponsoring me for the NumFOCUS John Hunter Technical Fellowship.

Thank you to Dr. Amy Pasquinelli for keeping the broader biological picture in mind, to Dr. Kun Zhang for knowing what's out there in the single-cell world, and to Dr. Sam Pfaff for neurobiological insights.

Thank you to Dr. Mike Lovci for teaching me about alternative splicing, to Dr. Boyko Kakaradov for teaching me about machine learning, to Gabriel Pratt for being my Python partner in crime, to Dr. Andrew Gross for answering my annoying \texttt{pandas} questions and to the entire Yeo Lab for being supportive of my open source endeavors.


Thank you especially to especially Pantea Khodami who will always me my Kappa Alpha Theta big sister and has always been there for me and has supported me every step of the way. Thank you to my friends Dr. Anne-Ruxandra Carvunis, Dr. Charisse Crenshaw, Colleen Stoyas, Cynthia Hsu, Kara Gordon, Emily Wheeler, David Nelles, Ron Batra, En-Ching Luo, Alain Domissy, Alice Ho, Rose Hurwitz, Alexa Robinson, Wendi Zhang, Nazita Lejevardi, Michelle Ma, Liam Fedus .... Thank you to the \emph{Lean In and Stay Classy San Diego crew}: Dianna Cowern, Shari Haynes, Kristen Pe\~na Breanna Berry, Daryl Fairweather, Shelly Wanamaker, and ... Thank you to my current and former roommates and affiliates Cailey Bromer, Daniella Bardalez Gagliuffi, Robert Schwartz, Jennifer Hammond, Patrick Wise, Sara Glass, Keawe Kolohe, Sharon and Matt Scott, and Andie Rotner for being a kind, supportive and welcoming environment to come home to, and reminding me there's so much more to life than just cells.

Thank you to my family, especially my mother and father for leaving the Soviet Union and bringing me to the United States to give me the opportunity to write a dissertation like this. Thank you to my brothers for being understanding of my mad scientist ways, thank you to my stepmother Ira whose family warmly accepted me as one of their own, especially Alya and Max Zolotorev.

Huge thank you to Kwasi Nti for surviving the Herculean task of dating a PhD student, for listening so hard and making sure I always made the best decision for me. I love you.

Chapter 1, in part, is currently being prepared for submission for publication of the material. Botvinnik, Olga; Song, Yan; Yeo, Gene W. The dissertation author was the primary investigator and author of this material. 
 
Chapter 2, in part, has been accepted for publication as the supplementary material as it may appear in Molecular Cell, 2017, Yan Song$^*$, Olga B Botvinnik$^*$, Michael T Lovci, Boyko Kakaradov, Patrick Liu, Jia L. Xu and Gene W Yeo ($^*$ These authors contributed equally to this work).  The dissertation author was one of the primary investigators and authors of this paper. 

Chapter 3, in full, has been accepted for publication as it may appear in Molecular Cell, 2017, Yan Song$^*$, Olga B Botvinnik$^*$, Michael T Lovci, Boyko Kakaradov, Patrick Liu, Jia L. Xu and Gene W Yeo ($^*$ These authors contributed equally to this work).  The dissertation author was one of the primary investigators and authors of this paper. 

\end{acknowledgements}


%% VITA
%
%  A brief vita is required in a doctoral thesis. See the OGS
%  Formatting Manual for more information.
%
\begin{vitapage}
\begin{vita}
  \item[2010] S.~B. in Mathematics, Massachusetts Institute of Technology
  \item[2010] S.~B. in Biological Engineering, Massachusetts Institute of Technology
  \item[2012] M.~S. in Biomolecular Engineering and Bioinformatics, University of California, Santa Cruz
  \item[2017] Ph.~D. in Bioinformatics and Systems Biology, University of California, San Diego
\end{vita}

\begin{publications}
\item Yan Song*, \textbf{Olga B Botvinnik}*, Michael T Lovci, Boyko Kakaradov, Patrick Liu, Jia L. Xu and Gene W Yeo. Single-cell alternative splicing analysis with Expedition reveals splicing dynamics during neuron differentiation. \emph{Accepted}. * These authors contributed equally to this work.

\item Curtis A Nutter, Elizabeth A Jaworski, Sunil K Verma, Vaibhav Deshmukh, Qiongling Wang, \textbf{Olga B Botvinnik}, Mario J Lozano, Ismail J Abass, Talha Ijaz, Allan R Brasier, Nisha J Garg, Xander H T Wehrens, Gene W Yeo, and Muge N Kuyumcu-Martinez. Dysregulation of RBFOX2 Is an Early Event in Cardiac Pathogenesis of Diabetes. \emph{Cell Reports}, 15(10):2200-2213, 2016.

\item Jong Wook Kim*, \textbf{Olga B Botvinnik}*, Omar Abudayyeh, Chet Birger, Joseph Rosen- bluh, Yashaswi Shrestha, Mohamed E Abazeed, Peter S Hammerman, Daniel DiCara, David J Konieczkowski, et al. Characterizing genomic alterations in cancer by complementary functional associations. \emph{Nature Biotechnology, 2016}. * These authors contributed equally to this work.

\item P Compeau and P Pevzner. \emph{Bioinformatics Algorithms} Volume 1, volume 1 of An Active Learning Approach. Active Learning Publishers LLC, 2 edition, 2015. Contributed text, figures, problems and code solutions, primarily to ``Chapter 4: How Do We Sequence Antibiotics?''.

\item Kris C Wood, David J Konieczkowski, Cory M Johannessen, Jesse S Boehm, Pablo Tamayo, \textbf{Olga B Botvinnik}, Jill P Mesirov, William C Hahn, David E Root, Levi A Garraway, et al. MicroSCALE screening reveals genetic modifiers of therapeutic response in melanoma. \emph{Science Signaling}, 5(224):rs4, 2012.

\item A Goncearenco, P Grynberg, \textbf{Olga B Botvinnik}, Geoff Macintyre, and Thomas Abeel. Highlights from the Eighth International Society for Computational Biology (ISCB) Student Council Symposium 2012. \emph{BMC Bioinformatics}, 2012.

\item Naomi Galili, Pablo Tamayo, \textbf{Olga B Botvinnik}, Jill P Mesirov, Margarita R Brooks, Gail Brown, and Azra Raza. Prediction of response to therapy with ezatiostat in lower risk myelodysplastic syndrome. \emph{Journal of Hematology \& Oncology}, 5(1):1, 2012.

\item Naomi Galili, Pablo Tamayo, \textbf{Olga B Botvinnik}, Jill P Mesirov, Jennifer Zikria, Gail Brown, and Azra Raza. Gene Expression Studies May Identify Lower Risk Myelodys- plastic Syndrome Patients Likely to Respond to Therapy with Ezatiostat Hydrochloride
(TLK199). \emph{Blood}, 118(21):2779-2779, 2011.

\item Michael F Berger, Gwenael Badis, Andrew R Gehrke, Shaheynoor Talukder, Anthony A Philippakis, Lourdes Pena-Castillo, Trevis M Alleyne, Sanie Mnaimneh, \textbf{Olga B Botvinnik}, Esther T Chan, et al. Variation in homeodomain DNA binding revealed by high-resolution analysis of sequence preferences. \emph{Cell}, 133(7):1266-1276, 2008.
\end{publications}

% \begin{publications}
%   \item Your Name, ``A Simple Proof Of The Riemann Hypothesis'', \emph{Annals of Math}, 314, 2007.
%   \item Your Name, Euclid, ``There Are Lots Of Prime Numbers'', \emph{Journal of Primes}, 1, 300 B.C.
% \end{publications}
\end{vitapage}


%% ABSTRACT
%
%  Doctoral dissertation abstracts should not exceed 350 words.
%   The abstract may continue to a second page if necessary.
%
\begin{abstract}
  Alternative splicing (AS) generates isoform diversity critical for cellular identity and homeostasis in multicellular life. Although AS variation has been observed among single cells for a few events, little is known about the biological significance of such variation. We developed Expedition, a computational framework consisting of outrigger, a \emph{de novo} splice graph transversal algorithm to detect AS; anchor, a Bayesian approach to assign modalities and bonvoyage, a visualization tool using non-negative matrix factorization to display modality changes. Applying Expedition to single iPSCs undergoing neuronal differentiation, we discover up to 20\% of AS exons exhibit bimodality and are flanked by more conserved introns harboring distinct cis-regulatory motifs. Bimodal exons constitute the majority of cell-type specific splicing, are highly dynamic during cellular transitions, preserve translatability and reveal intricacy of cell states invisible to global gene expression analysis. Systematic AS characterization in single cells redefines our understanding of AS complexity in cell biology.
\end{abstract}


\end{frontmatter}
